\documentclass{jsarticle}
%\documentclass{jsbook}
\usepackage[top=10truemm, bottom=15truemm, left=20truemm, right=20truemm]{geometry}
%\usepackage{geometry}
\usepackage{amssymb, amsmath}
\usepackage{listings,jvlisting,jlisting, color} % ソースコードの記述用
\usepackage{ascmac}
\usepackage{okumacro}
\usepackage{comment}

% 表紙に関する設定
\title{python基礎}
\date{} % 日付いらない
\author{}



\lstset{
	language=Python,    % 言語設定
	frame=single,       % 枠設定
	breaklines=true,    % 行が長くなった場合自動改行
	breakindent=10pt,   % 自動改行時のインデント
	columns=fixed,      % 文字の間隔を統一
	basewidth=0.5em,    % 文字の横のサイズを小さく
	numbers=left,       % 行数の位置
	numberstyle={\scriptsize \color{black}},  % 行数のフォント
	stepnumber=1,       % 行数の増間
	numbersep=1zw,      % 行数の余白
	xrightmargin=0zw,   % 左の余白
	xleftmargin=2zw,    % 右の余白
	keepspaces=true,    % スペースを省略せず保持
	lineskip=-0.2ex,    % 枠線の途切れ防止
	tabsize = 4,        % タブ数
	showstringspaces=false,  %文字列中の半角スペースを表示させない
	%%%%% style & color %%%%%
	basicstyle={\small\ttfamily \color{black}}, % 基礎の文字のフォントを設定
	identifierstyle={\small\ttfamily \color{black}},% 変数名のフォントを設定
	commentstyle={\small \color{black}},        % コメントのフォントを設定
	keywordstyle=[1]{\small\bfseries \color{black}}, % 予約語のフォントを設定
	keywordstyle=[2]{\small\bfseries \color{black}}, % 関数名のフォントを設定
	stringstyle={\small\ttfamily \color{black}},   % 文字列
	%%%%% style %%%%%
}

\renewcommand{\lstlistingname}{コード}

\begin{document}
\maketitle

\vspace{-20mm}

\section{変数}
\subsection{変数とは?}
変数とは「箱」のようなもの.数値や文字などを入れることができる.

\subsection{変数に使える名前}
変数の名前(変数名)のルールには次のものがある.

\begin{enumerate}
	\item{変数に使える文字は次の文字だけ}
	\begin{itemize}
		\item{小文字のアルファベット(a から z まで)}
		\item{大文字のアルファベット(A から Z まで)}
		\item{数字(0 から 9 まで)}
		\item{アンダーバー(アンダースコア)( \_ )} \par
		※変数名の先頭(1文字目)にアンダーバー( \_ ) を使用できるが \ruby{避}{さ}けた方が良い
	\end{itemize}
	\item{数字は変数名の先頭には使えない}
	\item{予約語(Pythonですでに使われている名前)を変数名にはできない}
	\begin{itembox}[l]{予約語の例}
		if, \; elif, \; else, \; False, \; True, \; try, \; for, \; continue, \; break, \; and, \; or, \; not \; など
	\end{itembox}

	\begin{figure}[htp]
		\begin{tabular}{cc} \hspace{3mm}
			\begin{minipage}[ht]{.48\textwidth}
				\begin{itembox}[l]{変数名として使える名前の例}
					a, \;abc, \;ABC,\; a1, \; xxx, \; abc\_xyz, \; \_a
				\end{itembox}
			\end{minipage} \hspace{5mm}
			\begin{minipage}[ht]{.4\textwidth}
				\begin{itembox}[l]{変数名として使えない名前の例}
					1a, \; 2b, \;  \$a, \; 1\_, \; if, \; else
				\end{itembox}
			\end{minipage}
		\end{tabular}
	\end{figure}
	\item{アルファベットの大文字小文字は別の変数として区別される} \par
	abc, \; ABC, \; Abc, \; aBC \; \ldots \, など同じ「エービーシー」でもすべて別の変数
\end{enumerate}

\subsection{代入}
\begin{figure}[htp]
	\begin{tabular}{cc}
		\begin{minipage}[ht]{.7\textwidth}
			変数には数値や文字を入れることができるが,そのことを{\textgt {代入}}という. \par
			{\large {=}} があるが数学とは\ruby{違}{ちが}い,右の例では右辺の値(この場合では 5)を \par
			左辺の変数(この場合は a)に代入することとなる. \par
			また,代入をしていない変数は中身がなく,この状態のことを\par
			{\textgt {未定義}}という.(右の例では,三行目の変数 c は未定義) \par
		\end{minipage}   \hspace{-4mm}
		\begin{minipage}[ht]{.25\textwidth}
			\begin{lstlisting}[caption=代入]
a = 5
b = "ABC"
c    \end{lstlisting}
		\end{minipage}
	\end{tabular}
\end{figure}
\newpage

\section{数式} \vspace{-5mm}
\begin{figure}[ht]
	\begin{tabular}{cc}
		\begin{minipage}{.6\textwidth}
			\subsection{使える記号} \vspace{-5mm}
			\begin{tabular}[t]{|c|c|c|c|c|}
				\hline
				記号 & 意味         & 例          & 結果 \\
				\hline \hline
				+    & 加算(足し算) & 5 + 8       & 13   \\ \hline
				-    & 減算(引き算) & 90 - 10     & 80   \\ \hline
				*    & 乗算(掛け算) & 4 * 7       & 28   \\ \hline
				/    & 除算(割り算) & 7 / 2       & 3.5  \\ \hline
				//   & 切り捨て除算 & 7 // 2      & 3    \\ \hline
				\%   & 剰余(あまり) & 7 \% 3      & 1    \\ \hline
				**   & 累乗         & 3 ** 4      & 81   \\ \hline
				()   & 括弧(かっこ) & (2 + 4) * 4 & 24   \\ \hline
			\end{tabular}
		\end{minipage}
		\hspace{-10mm}
		\begin{minipage}{.5\textwidth}
			\subsection{優先順位} \vspace{-4mm}
			\begin{tabular}[t]{|c|c|c|}
				\hline
				優先順位 & 記号 & 意味         \\
				\hline \hline
				1        & ()   & 括弧(かっこ) \\ \hline
				2        & **   & 累乗         \\ \hline
				3        & *    & 乗算(掛け算) \\ \hline
				         & /    & 除算(割り算) \\ \hline
				         & //   & 切り捨て除算 \\ \hline
				         & \%   & 剰余(あまり) \\ \hline
				4        & +    & 加算(足し算) \\ \hline
				         & -    & 減算(引き算) \\ \hline
			\end{tabular}
			\par
			※優先順位が同じ場合は左から順に計算される.
		\end{minipage}
	\end{tabular}
\end{figure}

\section{型(type)}
\subsection{データの型の種類} \vspace{-5mm}
\begin{figure}[h]
	\begin{tabular}{cc}
		\begin{minipage}[c]{.5\textwidth}
			\begin{tabular}[t]{|c|c|c|c|}
				\hline
				型名  & 意味   & 例                           \\
				\hline \hline
				int   & 整数   & 1,-12,2022,10,など       \\ \hline
				float & 小数   & 3.14...,0.5,12.53,など    \\ \hline
				str   & 文字列 & 'hello', "こんにちは",など \\ \hline
			\end{tabular}
		\end{minipage} \hspace{5mm}
		\begin{minipage}[c]{.6\textwidth}
			数学では整数は小数(実数)の中に含まれるが,\par コンピューターの世界では小数と整数の\ruby{扱}{あつか}いが \par
			異なるので明確に別物. \par
			文字列は'(シングルクォーテーション)  \par
			"(ダブルクォーテーション)記号で囲んだ部分のこと.
		\end{minipage}
	\end{tabular}
\end{figure}

\subsection{型と四則演算}
\subsubsection{int型とfloat型と四則演算}
数学の四則演算と同じ \par
int型とfloat型で四則演算を行うときは,int型をfloat型として型の\ruby{変換}{へんかん}がされ計算が行われる.

\subsubsection{str型と四則演算}
\begin{itemize}
	\item{str型では数学のような四則演算はできず,引き算・割り算はできない}
	\item{str型同士の足し算は文字列の結合(文字列をくっつける)をする} \par
	(掛け算と違いstr型とint型の足し算はできない) \par
	\item{str型とint型の掛け算は文字列をかけた分だけ\ruby{繰}{く}り返す} \par
	(足し算と違いstr型同士で掛け算はできない)
\end{itemize} \vspace{-5mm}
\begin{figure}[h]
	\begin{tabular}{cc}
		\begin{minipage}[t]{.4\textwidth}
			\begin{lstlisting}[caption=str型と四則演算]
# str型の四則演算 (足し算と掛け算)

a = "1"
b = "2"
ans = a + b  # str型の足し算
calc = a * 3 # str型の掛け算
print(ans)
print(calc)
\end{lstlisting}
		\end{minipage} \hspace{5mm}
		\begin{minipage}[t]{.6\textwidth}
			\begin{minipage}[t]{.2\textwidth}
				\begin{itembox}[l]{出力}
					12 \par
					111
				\end{itembox}
			\end{minipage}
			\begin{itembox}[l]{説明}
				str型の変数aとbがくっついたため出力が 12 となった \par
				a * 3でaを3回\ruby{繰}{く}り返すため,1が3回\ruby{繰}{く}り返された \par 文字列111となった
			\end{itembox}
		\end{minipage}
	\end{tabular}
\end{figure}

\newpage
\subsection{str型の数字} \vspace{-5mm}
\begin{figure}[h]
	\begin{tabular}{ccc}
		\begin{minipage}[t]{.45\textwidth}
			\begin{lstlisting}[caption=int型とstr型]
# int型とstr型の出力

int_val = -30    # int_valはint型(整数)
str_val = '-30'  # str_valはstr型(文字列)
print(int_val)
print(str_val)
\end{lstlisting}
		\end{minipage} \hspace{5mm}
		\begin{minipage}[t]{.1\textwidth}
			\begin{itembox}[l]{出力}
				-30 \par
				-30
			\end{itembox}
		\end{minipage} \hspace{5mm}
		\begin{minipage}[t]{.35\textwidth}
			\begin{itembox}[l]{説明}
				出力の見え方は全く同じでも,\par このプログラムにはint型か\par str型かという違いがある.
			\end{itembox}
		\end{minipage}
	\end{tabular}
\end{figure}


\section{if 文}
もし(条件式)ならば○○(処理)を実行する,というように条件に合った時だけ \par ○○(処理)を行うためのもの.
\subsection{if 文の書き方} \vspace{-5mm}
\begin{figure}[h]
	\begin{tabular}{cc}
		\begin{minipage}[c]{.3\textwidth}
			\begin{lstlisting}[caption=if 文の基本構文-その1]
if 条件式 :
  処理 \end{lstlisting}
		\end{minipage} \hspace{10mm}
		\begin{minipage}[c]{.7\textwidth}
			条件が一つのとき
		\end{minipage}
	\end{tabular}
\end{figure}
\vspace{-5mm}
\begin{figure}[h]
	\begin{tabular}{cc}
		\begin{minipage}[c]{.3\textwidth}
			\begin{lstlisting}[caption=if 文の基本構文-その2]
if 条件式 :
  処理1
else :
  処理2 \end{lstlisting}
		\end{minipage} \hspace{10mm}
		\begin{minipage}[c]{.5\textwidth}
			条件が一つとその条件以外のときに処理をしたいとき
		\end{minipage}
	\end{tabular}
\end{figure}
\vspace{-5mm}
\begin{figure}[h]
	\begin{tabular}{cc}
		\begin{minipage}[c]{.3\textwidth}
			\begin{lstlisting}[caption=if 文の基本構文-その3]
if 条件式1 :
  処理1
elif 条件式2 :
  処理2
else :
  処理3 \end{lstlisting}
		\end{minipage} \hspace{10mm}
		\begin{minipage}[c]{.5\textwidth}
			条件が複数のときとそれらの条件以外のときに処理を \par したいとき
		\end{minipage}
	\end{tabular}
\end{figure}

\subsection{条件式で使う記号}
\begin{figure}[h]
	\begin{tabular}{cc}
		\begin{minipage}[c]{.5\textwidth}
			\begin{tabular}[t]{|c|c|c|c|c|}
				\hline
				記号 & 意味       & 例       & 例の意味         \\
				\hline \hline
				$==$ & 等しい     & $x == 5$ & xは5と等しい     \\ \hline
				$!=$ & 等しくない & x $!=$ 3 & xは3と等しくない \\ \hline
				$>$  & より大きい & $5 > 2$  & 5は2より大きい   \\ \hline
				$<$  & より小さい & $2 < 3$  & 2は3より小さい   \\ \hline
				$>=$ & 以上       & $a >= 0$ & aは0以上         \\ \hline
				$<=$ & 以下       & $b <= 0$ & bは0以下         \\ \hline
			\end{tabular}
		\end{minipage}
		\begin{minipage}[c]{.465\textwidth}
			\begin{screen}
				等しい事を示す記号は,二つの等号($==$)が \par 使われている. \par
				一つの等号($=$) では代入になってしまうので注意 \par
				また,数学では以上・以下は $\geqq$・$\leqq$ と書くが,\par プログラミングではそのように書くことが \par できないため代わりに $>=$・$<=$ と書く.
			\end{screen}
		\end{minipage}
	\end{tabular}
\end{figure}
\newpage

\subsection{if 文の例} \vspace{-3mm}
\begin{figure}[ht]
	\begin{tabular}{cc}
		\begin{minipage}[c]{.45\textwidth}
			\begin{lstlisting}[caption=if 文の例1]
# 変数numの値が2の倍数かどうかの判定

num = 10
if num % 2 == 0 :
  print("num は2の倍数です.")
else :
  print("num は2の倍数ではありません.") \end{lstlisting}
		\end{minipage} \hspace{10mm}
		\begin{minipage}[c]{.46\textwidth}
			\begin{screen}
				numの値が2の倍数ならば, \par "num は2の倍数です." と表示. \par
				numの値が2の倍数でないならば, \par "num は2の倍数ではありません." と表示.\par
				({\textgt {numを2で割ったあまりが0}} [num \% 2 == 0] \par かどうかで 2の倍数かを判断する事が出来る)
			\end{screen}
		\end{minipage}
	\end{tabular}
\end{figure}
\vspace{-5mm}
\begin{figure}[h]
	\begin{tabular}{cc}
		\begin{minipage}[c]{.5\textwidth}
			\begin{lstlisting}[caption=if 文の例2]
# 変数numが正の数か負の数か0かどうかの判定

num = 0
if num > 0 :
  print("num は正の数です.")
elif num < 0 :
  print("num は負の数です.")
else :
  print("num は0です.") \end{lstlisting}
		\end{minipage} \hspace{10mm}
		\begin{minipage}[c]{.4\textwidth}
			\begin{screen}
				num が 0 より大きいなら (num $>$ 0) \par "num は正の数です." と表示. \par
				そうでないとき,num が 0 より小さいなら (num $<$ 0) "num は負の数です." と表示. \par
				それ以外のとき (つまりnum $==$ 0) \par "num は0です." と表示. \par
			\end{screen}
		\end{minipage}
	\end{tabular}
\end{figure}

\section{複雑なif 文}
\subsection{if 文の仕組み}
条件式からは True/False という値が返ってくる.これらは bool 型という「分類の値を持つ型」の値である. \par
条件式があっているときは True
条件式が間違っているときは False \par
if 文が実行されるときは条件式がTrueの値を持つときである.
\subsection{論理演算}
複数の条件があるときに使う.
\subsubsection{and(論理積)}
複数の条件がすべて合っていてほしいときに使う \par
すべての条件が合っているときその条件式全体はTrueとなる
\vspace{-5mm}
\begin{figure}[h]
	\begin{tabular}{cc}
		\begin{minipage}[t]{.4\textwidth}
			\begin{lstlisting}[caption=and]
# andの例

a = 1
b = 1
if (a == 1) and (b == 1) :
  print("Yes a == 1 and b == 1")
else :
  print("No  a == 1 and b == 1")

if (a == 1) and (b == 0) :
  print("Yes a == 1 and b == 0")
else :
  print("No  a == 1 and b == 0") \end{lstlisting}
		\end{minipage} \hspace{5mm}
		\begin{minipage}[t]{.6\textwidth}
			\begin{minipage}[t]{.5\textwidth}
				\begin{itembox}[l]{出力}
					Yes a == 1 and b == 1 \par
					No\; a == 1 and b == 0 \par
				\end{itembox}
			\end{minipage}
			\begin{itembox}[l]{説明}
				変数a, b の値は二つとも1である. \par
				1つ目のif 文の条件式は(a == 1),(b == 1) と2つとも \par 合っているためTrueとなりYes~が出力された. \par
				しかし,2つ目のif 文では (a == 1),(b == 0) とbの \par 条件式が合っていないためFalseとなりNo~が出力された. \par
			\end{itembox}

		\end{minipage} \hspace{5mm}
	\end{tabular}
\end{figure}
\newpage
\subsubsection{or(論理和)}
複数の条件のどれか一つでも合っていてほしいときに使う \par
どれか一つでも条件が合っているときその条件式全体はTrueとなる
\vspace{-5mm}
\begin{figure}[h]
	\begin{tabular}{cc}
		\begin{minipage}[t]{.4\textwidth}
			\begin{lstlisting}[caption=or]
# orの例

a = 1
b = 1
if (a == 1) or (b == 1) :
  print("Yes a == 1 or b == 1")
else :
  print("No  a == 1 or b == 1")

if (a == 1) or (b == 0) :
  print("Yes a == 1 or b == 0")
else :
  print("No  a == 1 or b == 0") \end{lstlisting}
		\end{minipage} \hspace{5mm}
		\begin{minipage}[t]{.6\textwidth}
			\begin{minipage}[t]{.5\textwidth}
				\begin{itembox}[l]{出力}
					Yes a == 1 or b == 1 \par
					Yes a == 1 or b == 0 \par
				\end{itembox}
			\end{minipage}
			\begin{itembox}[l]{説明}
				変数a, b の値は二つとも1である. \par
				1つ目のif 文の条件式は(a == 1),(b == 1) と2つとも \par 合っているためTrueとなりYes~が出力された.\par
				2つ目のif 文では (a == 1),(b == 0) とbの条件式が \par 合っていないがaの条件式が合っているためTrueとなり \par Yes~が出力された.
			\end{itembox}
		\end{minipage}
	\end{tabular}
\end{figure}

% \newpage

\subsubsection{not(否定)}
\vspace{-5mm}
\begin{figure}[h]
	\begin{tabular}{cc}
		\begin{minipage}[t]{.4\textwidth}
			\begin{lstlisting}[caption=not]
# notの例

a = 1
if not (a == 1) :
  print("Yes a == 1")
else :
  print("No  a == 1")

if not (a == 0) :
  print("Yes a == 0")
else :
  print("No  a == 0") \end{lstlisting}
		\end{minipage} \hspace{5mm}
		\begin{minipage}[t]{.6\textwidth}
			\begin{minipage}[t]{.3\textwidth}
				\begin{itembox}[l]{出力}
					No a == 1 \par
					Yes a == 0 \par
				\end{itembox}
			\end{minipage}
			\begin{itembox}[l]{説明}
				変数a の値は1である. \par
				1つ目のif 文の条件式は(a == 1)と合っているため \par Trueとなるが,それを否定しているためFalseとなり, \par No~が出力された.\par
				2つ目のif 文では (a == 0)と条件式が合っていないが, \par その条件式を否定しているためTrueとなりYes~が \par 出力された.
			\end{itembox}
		\end{minipage}
	\end{tabular}
\end{figure}

% \subsection{条件式の言い換え}
\subsection{計算の優先順位(再び)}
優先順位が高いほど先に計算が行われる. \par
同じ場合は左から右へ順に計算される. \par
\begin{tabular}[t]{|c|c|c|}
	\hline
	優先順位 &              & 同じ演算のときの優先順位                  \\
	\hline \hline
	高       & 括弧(かっこ) &                                           \\ \hline
	         & 算術演算     & 累乗(**)                                  \\ \hline
	         &              & 乗算(*),除算(/),切り捨て除算(//),剰余(\%) \\ \hline
	         &              & 加算(+),減算(-)                           \\ \hline
	         & 比較演算     & $>$,$>=$, $<$, $<=$                    \\ \hline
	低       & 論理演算     & not(否定)                                 \\ \hline
	         &              & and(論理積)                               \\ \hline
	         &              & or(論理和)                                \\ \hline
\end{tabular}

\section{for 文}
決められた回数\ruby{繰}{く}り返したいときに使うもの.\ruby{繰}{く}り返し文ともいう.
\subsection{for 文の書き方} \vspace{-5mm}
\begin{figure}[h]
	\begin{tabular}{cc}
		\begin{minipage}[c]{.4\textwidth}
			\begin{lstlisting}[caption=for 文の基本構文]
for 変数名 in 繰り返す回数 :
  繰り返したい処理1
  繰り返したい処理2 \end{lstlisting}
		\end{minipage} \hspace{10mm}
		\begin{minipage}[c]{.6\textwidth}
			for が書かれている行の最後には必ず\;{\textgt {:\;(コロン)}}を入れる. \par
			\ruby{繰}{く}り返したい処理の前にはタブを入れる \par
			(タブの有無で\ruby{繰}{く}り返す処理の範囲が変わってしまう)
		\end{minipage}
	\end{tabular}
\end{figure}

\subsection{for 文の例} \vspace{-5mm}
\begin{figure}[h]
	\begin{tabular}{ccc}
		\begin{minipage}[t]{.45\textwidth}
			\begin{lstlisting}[caption=for 文の例]
# 0から5までをすべて足したときの和

s = 0
for i in range(5) :
  print(i)    # 繰り返し部分
  s = s + i   # 繰り返し部分

print(s) # タブがないのでfor文の範囲外 \end{lstlisting}
		\end{minipage} \hspace{10mm}
		\begin{minipage}[t]{.1\textwidth}
			\begin{itembox}[l]{出力}
				0 \par
				1 \par
				2 \par
				3 \par
				4 \par
				10
			\end{itembox}
		\end{minipage} \hspace{10mm}
		\begin{minipage}[t]{.3\textwidth}
			\begin{itembox}[l]{説明}
				\ruby{繰}{く}り返し:1回目 \par
				\hspace{13mm}:2回目 \par
				\hspace{13mm}:3回目 \par
				\hspace{13mm}:4回目 \par
				\hspace{13mm}:5回目 \par
				8行目のprint関数の出力
			\end{itembox}
		\end{minipage}
	\end{tabular}
\end{figure}


\section{while 文}
for 文と同じく\ruby{繰}{く}り返しが必要な処理を行うために使うもの. \par
条件式がTrueの間\ruby{繰}{く}り返しが続き,Falseのとき\ruby{繰}{く}り返しが終了する.
\subsection{while 文の書き方}
\begin{figure}[h]
	\begin{tabular}{cc}
		\begin{minipage}[c]{.45\textwidth}
			\begin{lstlisting}[caption=while 文の基本構文]
while 条件式 :
  条件式が'True'のとき繰り返したい処理\end{lstlisting}
		\end{minipage} \hspace{3mm}
		\begin{minipage}[c]{.55\textwidth}
			\begin{lstlisting}[caption=while 文の注意点]
while True : # もしくは条件式がずっとTrue
繰り返したい処理  # 処理が永遠に繰り返される \end{lstlisting}
		\end{minipage}
	\end{tabular}
\end{figure}
while が書かれている行の最後には必ず\;{\textgt {:\;(コロン)}}を入れる. \par
while 文は条件式がTrueのとき処理を\ruby{繰}{く}り返すので,「while True:」のときは永遠に処理を\ruby{繰}{く}り返す.
また,指定した条件式を間違えたり,条件式で変数を使っているときに\ruby{繰}{く}り返し処理のなかで変数を条件式を満たすように変化させなかった場合も条件式の結果がTrueだと永遠に処理を\ruby{繰}{く}り返してしまう.
\newpage
\subsection{break}
for 文や while 文の\ruby{繰}{く}り返し処理の中で\ruby{繰}{く}り返しを中断したいときに使う.\par
break を使うとfor 文や while 文の\ruby{繰}{く}り返し処理自体が終了する.
\begin{figure}[h]
	\begin{tabular}{ccc}
		\begin{minipage}[t]{.45\textwidth}
			\begin{lstlisting}[caption=break の使い方]
# breakの使い方

n = 0
while True :
  n = n + 1
  print(f'n = {n}')
  if n == 5:
    break    \end{lstlisting}
		\end{minipage} \hspace{5mm}
		\begin{minipage}[t]{.1\textwidth}
			\begin{itembox}[l]{出力}
				n = 1 \par
				n = 2 \par
				n = 3 \par
				n = 4 \par
				n = 5 \par
			\end{itembox}
		\end{minipage} \hspace{5mm}
		\begin{minipage}[t]{.4\textwidth}
			\begin{itembox}[l]{説明}
				while Trueで永遠に\ruby{繰}{く}り返しが実行 \par されそうだが,
				n が 5 のときに breakによってループが終了したため,出力が \par 5まで表示された. \par
			\end{itembox}
		\end{minipage}
	\end{tabular}
\end{figure}

\subsection{continue}
\ruby{繰}{く}り返し処理の中で特定の処理のときだけ,その処理をスキップする場合に使う. \par
continue を使ってもfor 文やwhile 文の\ruby{繰}{く}り返し自体は終了しない.
\begin{figure}[h]
	\begin{tabular}{ccc}
		\begin{minipage}[t]{.45\textwidth}
			\begin{lstlisting}[caption=continue の使い方]
# continueの使い方

n = 0
while n < 5 :
  n = n + 1
  if n == 3:
    continue
  print(f'n = {n}')  \end{lstlisting}
		\end{minipage} \hspace{5mm}
		\begin{minipage}[t]{.1\textwidth}
			\begin{itembox}[l]{出力}
				n = 1 \par
				n = 2 \par
				n = 4 \par
				n = 5 \par
			\end{itembox}
		\end{minipage} \hspace{5mm}
		\begin{minipage}[t]{.4\textwidth}
			\begin{itembox}[l]{説明}
				n が 3 のときに continueによって\par ループがスキップされたため,\par
				出力が3のときだけ結果が表示され \par なかった.
			\end{itembox}
		\end{minipage}
	\end{tabular}
\end{figure}

\subsection{条件式とTrueとbreak}
\begin{figure}[h]
	\begin{tabular}{cc}
		\begin{minipage}[c]{.45\textwidth}
			\begin{lstlisting}[caption=条件式による繰り返しの終了]
n = 0
while n < 5 :
  n = n + 1
  print(f'n = {n}') \end{lstlisting}
		\end{minipage} \hspace{3mm}
		\begin{minipage}[c]{.55\textwidth}
			\begin{lstlisting}[caption=break による繰り返しの終了]
n = 0
while True :
  n = n + 1
  print(f'n = {n}')
  if n == 5:
    break    \end{lstlisting}
		\end{minipage}
	\end{tabular}
\end{figure}
\begin{screen}
	whileの条件式による終了か break による終了かの違いだけで,どちらも,n が 5 のときに\ruby{繰}{く}り返し処理が \par
	終了する同じ処理である.ただし,breakでの処理の方では,breakを忘れてしまうと処理が永遠に\ruby{繰}{く}り返す\par
	ので注意が必要となる.
\end{screen}
\newpage
\section{関数-その1}
\begin{figure}[h]
	\begin{tabular}{cc}
		\begin{minipage}{.6\textwidth}
			\subsection{関数とは}
			関数とは,値を入れる(入れなくてもよい)と決められた処理を\par
			実行してその結果を返すプログラムの部品のようなもの.\par
			入れる値のことを{\textgt {引数(ひきすう)}}と言い, 値を処理して\par 返ってきた結果を
			{\textgt {返り値/戻り値(かえりち/もどりち)}}と言う. \par
			関数にはPythonにすでにあるものや自分で作れるものがある. \par

			関数の引数に関数を入れたり,関数の返り値を変数に \par 代入することが出来る. \par
		\end{minipage}
		\hspace{10mm}
		\begin{minipage}{.3\textwidth}
			\vspace{-12mm}
			\subsection{関数を使うときの例}
			\begin{lstlisting}[caption=関数]
# 関数の使い方

関数名(引数) \end{lstlisting}
		\end{minipage}
	\end{tabular}
\end{figure}

\subsection{print関数}
print関数:文字列や数値の計算結果を出力する関数 \par
引数:文字列や数値,計算結果
\subsubsection{print関数の使い方} \vspace{-5mm}
\begin{figure}[h]
	\begin{tabular}{ccc}
		\begin{minipage}[t]{.4\textwidth}
			\begin{lstlisting}[caption=print関数]
# print関数の使い方

print("こんにちは")
print(1000)
a = 10
b = 30
print(a)
print(b)
print(a + b) \end{lstlisting}
		\end{minipage} \hspace{5mm}
		\begin{minipage}[t]{.15\textwidth}
			\begin{itembox}[l]{出力}
				こんにちは \par
				1000 \par
				10 \par
				30 \par
				40 \par
			\end{itembox}
		\end{minipage} \hspace{5mm}
		\begin{minipage}[t]{.3\textwidth}
			\begin{itembox}[l]{説明}
				()の中の文字列や数値,\par 変数の値,計算式の結果を \par
				出力する
			\end{itembox}
		\end{minipage}
	\end{tabular}
\end{figure}
\subsubsection{print関数で変数と文字をを一緒に使うには?} \vspace{-5mm}
\begin{figure}[h]
	\begin{tabular}{cc}
		\begin{minipage}[t]{.4\textwidth}
			\begin{lstlisting}[caption=print関数]
# print関数で変数と文字を一緒に使う

name = "タロウ"
print(f'{name}さん,こんにちは') \end{lstlisting}
		\end{minipage} \hspace{5mm}
		\begin{minipage}[t]{.5\textwidth}
			\begin{minipage}[t]{.6\textwidth}
				\begin{itembox}[l]{出力}
					タロウさん,こんにちは
				\end{itembox}
			\end{minipage}
			\begin{itembox}[l]{説明}
				文字列をf''で囲み,変数を\{\}で囲むことで \par
				変数の値を文字列と一緒に出力することができる.
			\end{itembox}
		\end{minipage}
	\end{tabular}
\end{figure}

\newpage
\subsection{len関数}
引数の値の長さを返す関数 \vspace{-5mm}
\begin{figure}[h]
	\begin{tabular}{cc}
		\begin{minipage}[t]{.4\textwidth}
			\begin{lstlisting}[caption=len関数]
# len関数の使い方

s = "abcde"
l = len(s)
print(l) \end{lstlisting}
		\end{minipage} \hspace{5mm}
		\begin{minipage}[t]{.4\textwidth}
			\begin{minipage}[t]{.25\textwidth}
				\begin{itembox}[l]{出力}
					5
				\end{itembox}
			\end{minipage}
			\begin{itembox}[l]{説明}
				変数s の値 "abcde" の長さが5 のため, \par
				len関数の戻り値は5 となる
			\end{itembox}
		\end{minipage}
	\end{tabular}
\end{figure}


% \newpage
\subsection{range関数}
一定の間隔での数値の並びを生成する関数 \par
range関数の引数の与え方は,range(開始値, 終了値, ステップ幅)ある. \par
開始値とステップ幅はそれぞれ省略でき,省略した場合は(開始値:0, ステップ幅:1)が初期値として  \par
与えられる.生成できる値は開始値から{\textgt{終了値より手前の整数}}までの範囲である.
\vspace{-5mm}
\begin{figure}[h]
	\begin{tabular}{cc}
		\begin{minipage}[t]{.4\textwidth}
			\begin{lstlisting}[caption=range関数]
# range関数

for i in range(5) :
  # 0, 1, 2, 3, 4

for i in range(5, 10) :
  # 5, 6, 7, 8, 9

for i in range(0, 10, 2) :
  # 0, 2, 4, 6, 8
\end{lstlisting}
		\end{minipage} \hspace{5mm}
		\begin{minipage}[t]{.6\textwidth}
			\begin{itembox}[l]{説明}
				3 行目では開始値,ステップ幅が省略されているため,0から始まり,1ずつ上がっていく\par
				6 行目では5から始まり,9まで上がっている\par
				9 行目ではステップ幅が2のため,2ずつ上がり,終了値10,つまり9までの範囲で値が生成される
			\end{itembox}
		\end{minipage}
	\end{tabular}
\end{figure}

\subsection{max, min関数}
\subsubsection{max関数}
2つ以上の引数でその中から最大の値を返す関数 \par
引数:数値,文字列,リスト \hspace{5mm}
返り値:最大値
\vspace{-5mm}
\begin{figure}[h]
	\begin{tabular}{ccc}
		\begin{minipage}[t]{.4\textwidth}
			\begin{lstlisting}[caption=max関数]
# max関数の使い方

max_num = max(1, 3, 4, 2)
max_str = max("abc", "xyz", "ijk")
print(max_num)
print(max_str) \end{lstlisting}
		\end{minipage} \hspace{5mm}
		\begin{minipage}[t]{.1\textwidth}
			\begin{itembox}[l]{出力}
				4 \par
				xyz \par
			\end{itembox}
		\end{minipage} \hspace{5mm}
		\begin{minipage}[t]{.45\textwidth}
			\begin{itembox}[l]{説明}
				max\_numには(1, 3, 4, 2)の中で\par 一番大きい値である4が代入される \par
				max\_strには("abc", "xyz", "ijk")の中で \par 一番大きい値であるxyzが代入される
			\end{itembox}
		\end{minipage}
	\end{tabular}
\end{figure}
\newpage
\subsubsection{min関数}
2つ以上の引数でその中から最小の値を返す関数 \par
引数:数値,文字列,リスト \hspace{5mm}
返り値:最小値
\vspace{-5mm}
\begin{figure}[h]
	\begin{tabular}{ccc}
		\begin{minipage}[t]{.4\textwidth}
			\begin{lstlisting}[caption=min関数]
# min関数の使い方

min_num = min(1, 3, 4, 2)
min_str = min("abc", "xyz", "ijk")
print(min_num)
print(min_str) \end{lstlisting}
		\end{minipage} \hspace{5mm}
		\begin{minipage}[t]{.1\textwidth}
			\begin{itembox}[l]{出力}
				1 \par
				abc \par
			\end{itembox}
		\end{minipage} \hspace{5mm}
		\begin{minipage}[t]{.45\textwidth}
			\begin{itembox}[l]{説明}
				min\_numには(1, 3, 4, 2)の中で\par 一番小さい値である1が代入される \par
				min\_strには("abc", "xyz", "ijk")の中で\par 一番小さい値であるabcが代入される \par
			\end{itembox}
		\end{minipage}
	\end{tabular}
\end{figure}

% \newpage

\subsection{input関数}
変数に任意(好き)な文字列を代入する事が出来る関数 \par
引数:なし
\vspace{-5mm}
\begin{figure}[h]
	\begin{tabular}{ccc}
		\begin{minipage}[t]{.4\textwidth}
			\begin{lstlisting}[caption=input関数]
# input関数の使い方

value = input()
print(value) \end{lstlisting}
		\end{minipage} \hspace{5mm}
		\begin{minipage}[t]{.2\textwidth}
			\begin{itembox}[l]{入力}
				abcdef
			\end{itembox}
			\begin{itembox}[l]{出力}
				abcdef
			\end{itembox}
		\end{minipage} \hspace{5mm}
		\begin{minipage}[t]{.35\textwidth}
			\begin{itembox}[l]{説明}
				入力したabcdefがvalueに \par 代入され,print関数で出力される
			\end{itembox}
		\end{minipage}
	\end{tabular}
\end{figure}

\subsection{type関数}
データの種類を調べる関数 \par
引数:データ \hspace{5mm}
返り値:種類名
\vspace{-5mm}
\begin{figure}[h]
	\begin{tabular}{cc}
		\begin{minipage}[t]{.4\textwidth}
			\begin{lstlisting}[caption=type関数]
# type関数の使い方

int_val = 3
float_val = 3.14
print(type(int_val))
print(type(float_val)) \end{lstlisting}
		\end{minipage} \hspace{5mm}
		\begin{minipage}[t]{.5\textwidth}
			\begin{minipage}[t]{.4\textwidth}
				\begin{itembox}[l]{出力}
					$<$class 'int'$>$ \par
					$<$class 'float'$>$ \par
				\end{itembox}
			\end{minipage}
			\begin{itembox}[l]{説明}
				int\_valの値はint型,float\_valの値はfloat型と \par データの種類が出力された
			\end{itembox}
		\end{minipage} \hspace{5mm}
	\end{tabular}
\end{figure}

\newpage
\subsection{int, float, str関数}
\subsubsection{int関数}
引数に与えられた値をint型に\ruby{変換}{へんかん}する関数 \vspace{-5mm}
\begin{figure}[h]
	\begin{tabular}{cc}
		\begin{minipage}[t]{.4\textwidth}
			\begin{lstlisting}[caption=int関数]
# int関数の使い方

float_val = 3.1415
int_val   = int(float_val)
print(int_val)
print(type(int_val)) \end{lstlisting}
		\end{minipage} \hspace{5mm}
		\begin{minipage}[t]{.6\textwidth}
			\begin{minipage}[t]{.3\textwidth}
				\begin{itembox}[l]{出力}
					3 \par
					$<$class 'int'$>$ \par
				\end{itembox}
			\end{minipage}
			\begin{itembox}[l]{説明}
				int型は整数型のため小数が整数に\ruby{変換}{へんかん}され, 3.1415 が 3 と \par なった.
				type関数の返り値からもint型になったことが分かる.
			\end{itembox}
		\end{minipage}
	\end{tabular}
\end{figure}

\subsubsection{float関数}
引数に与えられた値をfloat型に\ruby{変換}{へんかん}する関数 \vspace{-5mm}
\begin{figure}[h]
	\begin{tabular}{cc}
		\begin{minipage}[t]{.4\textwidth}
			\begin{lstlisting}[caption=float関数]
# float関数の使い方

int_val   = 100
float_val = float(int_val)
print(float_val)
print(type(float_val)) \end{lstlisting}
		\end{minipage} \hspace{5mm}
		\begin{minipage}[t]{.6\textwidth}
			\begin{minipage}[t]{.3\textwidth}
				\begin{itembox}[l]{出力}
					100.0 \par
					$<$class 'float'$>$ \par
				\end{itembox}
			\end{minipage}
			\begin{itembox}[l]{説明}
				float型は小数の型のため整数が小数に\ruby{変換}{へんかん}され,100 が \par 100.0 となった.
				type関数の返り値からもfloat型に \par なったことが分かる.
			\end{itembox}
		\end{minipage}
	\end{tabular}
\end{figure}

\subsubsection{str関数}
引数に与えられた値をstr型に\ruby{変換}{へんかん}する関数 \vspace{-5mm}
\begin{figure}[h]
	\begin{tabular}{cc}
		\begin{minipage}[t]{.4\textwidth}
			\begin{lstlisting}[caption=str関数]
# str関数の使い方

int_val = 100
str_val = str(int_val)
print(str_val)
print(type(str_val)) \end{lstlisting}
		\end{minipage} \hspace{5mm}
		\begin{minipage}[t]{.6\textwidth}
			\begin{minipage}[t]{.3\textwidth}
				\begin{itembox}[l]{出力}
					100 \par
					$<$class 'str'$>$ \par\par
				\end{itembox}
			\end{minipage}
			\begin{itembox}[l]{説明}
				見た目ではわからないがint型からstr型に\ruby{変換}{へんかん}されたことが \par
				type関数によってわかる
			\end{itembox}
		\end{minipage}
	\end{tabular}
\end{figure}

\newpage
\section{list (リスト)}
\subsection{リストとは}
複数のデータに順番を付けて一緒のものとしたいときに利用するもの.
\subsection{リストの生成} \vspace{-10mm}
\begin{figure}[h]
	\begin{tabular}{cc}
		\begin{minipage}[t]{.4\textwidth}
			\begin{lstlisting}[caption=リストの生成]
# リストの生成

a = [5, 1, 3, 4]
b = ['abc', 'def', 'ghi']

t = 5
c = [t, 1, 3, 4]
\end{lstlisting}
		\end{minipage} \hspace{5mm}
		\begin{minipage}[t]{.6\textwidth}
			\begin{itembox}[l]{説明}
				リストの生成には [ ] の中に {\LARGE ,} で区切り具体的にリストの \par
				中身を書くことで生成できる.\par
				6,7 行目のように変数を入れてもよい
			\end{itembox}
		\end{minipage}
	\end{tabular}
\end{figure}


%\newpage
\subsection{リストの使い方}
リストの個々のデータのことを要素という言い,要素の順番を指定する値のことを添え字という言い方をする. \par
また,要素の順番は 0 から始まる.
\subsubsection{要素へのアクセス方法} \vspace{-5mm}
\begin{figure}[h]
	\begin{tabular}{cc}
		\begin{minipage}[t]{.4\textwidth}
			\begin{lstlisting}[caption=要素へのアクセス方法]
# リストの使い方
# (要素へのアクセス方法)

#    要素の順番
#    0  1  2  3
a = [5, 1, 3, 4]
print(a[0])

# 正の添え字と負の添え字
print(a[3], a[-1]) \end{lstlisting}
		\end{minipage} \hspace{5mm}
		\begin{minipage}[t]{.6\textwidth}
			\begin{minipage}[t]{.3\textwidth}
				\begin{itembox}[l]{出力}
					5 \par
					4 4 \par
				\end{itembox}
			\end{minipage}
			\begin{itembox}[l]{説明}
				リストの 0 番目である値 5 が出力される. \par
				添え字には負の値を指定することができ, 負の値を
				指定した\par 場合はリストの後ろの要素から指定されていく. \par
			\end{itembox}
		\end{minipage}
	\end{tabular}
\end{figure}

\begin{figure}[h]
	\begin{tabular}{cc}
		\begin{minipage}[t]{.4\textwidth}
			\begin{lstlisting}[caption=要素へのアクセス方法-その2]
# リストの使い方

#    要素の順番
#    0  1  2  3
a = [5, 1, 3, 4]
print(a[0:3]) \end{lstlisting}
		\end{minipage} \hspace{5mm}
		\begin{minipage}[t]{.6\textwidth}
			\begin{minipage}[t]{.3\textwidth}
				\begin{itembox}[l]{出力}
					[5, 1, 3]
				\end{itembox}
			\end{minipage}
			\begin{itembox}[l]{説明}
				添え字に[先頭番号:終了番号]を与えると,リストの一部だけを取り出すことができる.\par
				このとき,終了番号の手前までの値が取り出せることに注意する. \par
				このような添字の指定の仕方をスライスという \par
			\end{itembox}
		\end{minipage}
	\end{tabular}
\end{figure}

\newpage
\subsubsection{リストの基本操作}
\begin{figure}[h]
	\begin{tabular}{cc}
		\begin{minipage}[t]{.4\textwidth}
			\begin{lstlisting}[caption=リストに値を追加]
# リストに値を追加

a = [5, 1, 3, 4]
a.append(2)
print(a) \end{lstlisting}
		\end{minipage} \hspace{5mm}
		\begin{minipage}[t]{.6\textwidth}
			\begin{minipage}[t]{.3\textwidth}
				\begin{itembox}[l]{出力}
					[5, 1, 4, 4, 2]
				\end{itembox}
			\end{minipage}
			\begin{itembox}[l]{説明}
				リストa の最後に 2 が追加された. \par
				append()を使うとき()の中は{\textgt{追加したい値}}を書く.
			\end{itembox}
		\end{minipage}
	\end{tabular}
\end{figure}

\begin{figure}[h]
	\begin{tabular}{cc}
		\begin{minipage}[t]{.4\textwidth}
			\begin{lstlisting}[caption=リストの値を削除]
# リストの値を削除

a = [5, 1, 3, 4]
a.pop(0)
print(a) \end{lstlisting}
		\end{minipage} \hspace{5mm}
		\begin{minipage}[t]{.6\textwidth}
			\begin{minipage}[t]{.3\textwidth}
				\begin{itembox}[l]{出力}
					[1, 4, 4, 2]
				\end{itembox}
			\end{minipage}
			\begin{itembox}[l]{説明}
				リストの 0 番目である値 5 がリストa から削除された. \par
				pop()を使うときは()の中は,{\textgt{削除したい値の添え字}}にする. \par
			\end{itembox}
		\end{minipage}
	\end{tabular}
\end{figure}


\section{関数-その2}
\subsection{自分で関数を作ってみよう}
\begin{figure}[h]
	\begin{tabular}{c}
		\begin{minipage}[t]{.4\textwidth}
			\begin{lstlisting}[caption=関数定義の形]
# 関数定義の形

def 関数名(引数) :
  実行したいプログラム
  return 返り値
\end{lstlisting}
		\end{minipage}
	\end{tabular}
\end{figure}



\begin{itembox}[l]{関数を作るときのルール}
	\begin{itemize}
		\item{複数の関数を作るときは,関数名が被らないように注意をすること.\par
		            関数名のルールは変数名の時と同じ.}
		\item{引数は2つ以上でも良く,そのときは {\textbf {,}}で引数と引数を区切る.\par
		また,引数はなくても良い.しかし,その場合でも関数名の後ろには()をつけなければいけない}
		\item{return 文は特定の条件が成立した場合の if 文の中など,関数定義の最後
		            以外の場所に書いてもよい.いらない場合は書かなくてもよい.}
		\item{関数の返り値には計算式を書いても良い. \par}
		\item{関数を作る時の def 後の( )内の引数名と実際に呼び出して使う時の引数名
		            は一致している必要はない.}
	\end{itemize}
\end{itembox}

\newpage

\subsection{変数が使える範囲(スコープ)}
変数が使える有効な範囲のことをスコープという. \par
大きく分けて関数内で使える範囲と,プログラム全体で使える範囲がある.
\subsubsection{グローバル変数とローカル変数}
\begin{itemize}
	\item {ローカル変数 :限られた範囲で使われる変数} \par
	      関数内で定義された(代入された)変数は関数内でのみ利用可能で,関数の実行ごとに関数の実行が終了すれば失われる

	\item {グローバル変数:プログラム全体で使える変数} \par
	      関数外で定義されている変数は値を読み取ることのみ可能
	      関数内で global 宣言された変数のみ,グローバル変数に代入可能.
	      このglobal宣言をせずに代入をするとグローバル変数がローカル変数になってしまう.
\end{itemize}

% \newpage

\section{モジュール}
\subsection{モジュールのとは?}
拡張機能のこと.Pythonでは様々なモジュールをインポートして利用する. \par
モジュール自体はPython のプロジェクトファイルでしかない.
\subsection{モジュールのインポート方法}
モジュール名のインポート方法は大きく分けて以下の2つ
\begin{itemize}
	\item{import モジュール名}
	\item{from モジュール名 import 機能名}
\end{itemize}
\subsubsection{import と from について}
\begin{itemize}
	\item{import~} \par
	モジュール全体を利用する,という宣言 \par
	モジュール名の関数を使うときは,関数の前にモジュール名を付ける必要がある.
	\item{from~} \par
	モジュールの一部を利用する,という宣言 \par
	1つ1つ呼び出す代わりに関数を使うときに関数の前にモジュール名は付けない \par
	from モジュール名 import * はそのモジュールの関数や変数をすべて直接呼び出している.
\end{itemize}

\subsection{import~と from~の使い分け}
基本は自由.自分やチームのルールがある場合はそれに従って書く.\par
変数名や関数名はかぶる可能性が有るため,少数しかモジュールや関数を使わないなら from~を \par
たくさんのモジュールを使うときは import~を使う方がいい場合が多い.

\newpage
\subsection{インポートの仕方}
\begin{figure}[h]
	\begin{tabular}{cc}
		\begin{minipage}[t]{.45\textwidth}
			自作のモジュールを使うときは,モジュールの\par
			ファイルが使いたいプログラムのファイルと \par
			同じ場所にあり, モジュール名と同じファイルが\par ないように気を付ける. \par
			右のような自作モジュール(sample.py)があると \par する.
			このときの,インポート方法を紹介する.
		\end{minipage} \hspace{15mm}
		\begin{minipage}[t]{.4\textwidth}
			\vspace{-10mm}
			\begin{lstlisting}[caption=sample.py]
hello = "こんにちは"
color = ["red", "green", "黒"]

def right(var_list) :
  ret = var_list[-1]
  print(ret)
  return ret \end{lstlisting}
		\end{minipage}
	\end{tabular}
\end{figure}


\begin{figure}[h]
	\begin{tabular}{cc}
		\begin{minipage}[t]{.4\textwidth}
			\begin{lstlisting}[caption=インポートの仕方-その1]
import sample

sample.color
sample.hello \end{lstlisting}
		\end{minipage} \hspace{5mm}
		\begin{minipage}[t]{.6\textwidth}
			\begin{screen}
				import モジュール名 \par
				のときは モジュール名. というのを頭につけて変数や関数の \par 名前を書く.
				今回であれば sample. をつける必要がある.
			\end{screen}
		\end{minipage}
	\end{tabular}
\end{figure}

\begin{figure}[h]
	\begin{tabular}{cc}
		\begin{minipage}[t]{.4\textwidth}
			\begin{lstlisting}[caption=インポートの仕方-その2]
import sample as sp

sp.color
sp.hello \end{lstlisting}
		\end{minipage} \hspace{5mm}
		\begin{minipage}[t]{.6\textwidth}
			\begin{screen}
				import モジュール名 as 別の名前 \par
				のときはモジュール名を別の名前に変更して,変更した名前を\par 頭につけて変数や関数の名前を書く. \par
				今回であれば sample. ではなく sp. をつける必要がある. \par
				この方法はモジュール名が長いものを省略して書きたいときによく使われる方法である.
			\end{screen}
		\end{minipage}
	\end{tabular}
\end{figure}

\begin{figure}[h]
	\begin{tabular}{cc}
		\begin{minipage}[t]{.4\textwidth}
			\begin{lstlisting}[caption=インポートの仕方-その3]
from sample import color, right

color
hello \end{lstlisting}
		\end{minipage} \hspace{5mm}
		\begin{minipage}[t]{.6\textwidth}
			\begin{screen}
				from モジュール名 import モジュールの関数や変数(複数可) \par
				のときは モジュール名. というのを頭につけて関数の名前や\par 変数を書く必要がなくなる.\par
				しかし,宣言していない関数や変数は使えないため注意\par (今回では hello は宣言していないため使えない)
			\end{screen}
		\end{minipage}
	\end{tabular}
\end{figure}

\begin{figure}[h]
	\begin{tabular}{cc}
		\begin{minipage}[t]{.4\textwidth}
			\begin{lstlisting}[caption=インポートの仕方-その4]
from sample import *

color
hello \end{lstlisting}
		\end{minipage} \hspace{5mm}
		\begin{minipage}[t]{.6\textwidth}
			\begin{screen}
				from モジュール名 import * \par
				のときはモジュールの関数や変数を import の後ろに\par 書かなくてもモジュールの関数や変数が使えるようになる.
			\end{screen}
		\end{minipage}
	\end{tabular}
\end{figure}


% \section{turtle}
% \section{tkiner}


\end{document}