\documentclass[dvipdfmx]{jsbook}
\usepackage{amsmath,amssymb,amsthm}
\usepackage{listings,jvlisting,jlisting, color} % ソースコードの記述用
\usepackage[bottom=15truemm, left=20truemm, right=20truemm]{geometry}	% 余白の設定
\usepackage{okumacro} % ルビ用
\usepackage{tcolorbox, ascmac} % 枠つけ用
\pagestyle{headings} % ヘッダー

\newcommand{\countup}[1]{\setcounter{chapter}{#1} \setcounter{section}{0}}
\newcommand{\info}[2]{\begin{tcolorbox}[colframe=gray, colback=black!10!white, coltitle=white, fonttitle=\bfseries, title={#1}]
{#2}\end{tcolorbox}}

\lstset{
	language=Python,    % 言語設定
	frame=single,       % 枠設定
	breaklines=true,    % 行が長くなった場合自動改行
	breakindent=10pt,   % 自動改行時のインデント
	columns=fixed,      % 文字の間隔を統一
	basewidth=0.5em,    % 文字の横のサイズを小さく
	numbers=left,       % 行数の位置
	numberstyle={\scriptsize \color{black}},  % 行数のフォント
	stepnumber=1,       % 行数の増間
	numbersep=1zw,      % 行数の余白
	xrightmargin=0zw,   % 左の余白
	xleftmargin=2zw,    % 右の余白
	keepspaces=true,    % スペースを省略せず保持
	lineskip=-0.2ex,    % 枠線の途切れ防止
	tabsize = 4,        % タブ数
	showstringspaces=false,  %文字列中の半角スペースを表示させない
	%%%%% style & color %%%%%
	basicstyle={\small\ttfamily \color{black}}, % 基礎の文字のフォントを設定
	identifierstyle={\small\ttfamily \color{black}},% 変数名のフォントを設定
	commentstyle={\small \color{black}},        % コメントのフォントを設定
	keywordstyle=[1]{\small\bfseries \color{black}}, % 予約語のフォントを設定
	keywordstyle=[2]{\small\bfseries \color{black}}, % 関数名のフォントを設定
	stringstyle={\small\ttfamily \color{black}},   % 文字列
	%%%%% style %%%%%
}

\renewcommand{\lstlistingname}{コード}

%% 表紙に関する設定
%\title{python基礎}
%\date{\today} % 日付いらない
%\author{}


\begin{document}
%\maketitle

%% 目次の出力
%\tableofcontents
%\clearpage

\section*{変数}
\countup{1}
\section{変数とは?}
変数とは数値や文字などを入れることができる入れ物.
\section{変数の名前}
変数名(変数の名前)にはルールがある.
\begin{enumerate}
	\item 変数に使える文字は次の文字だけ
	      \begin{itemize}
		      \item アルファベット (a $\sim$ z, A $\sim$ Z)
		      \item 数字(0 $\sim$ 9)
		      \item アンダースコア(\_) \par
		            ※変数名の先頭(1文字目)にアンダーバー(\_) を使用できるが \ruby{避}{さ}けた方が良い
	      \end{itemize}
	\item 数字は変数名の先頭には使えない
	\item 予約語(Pythonですでに使われている名前)は変数名にはできない
	      \info{予約語の例}{if, \; elif, \; else, \; False, \; True, \; try, \; for, \; continue, \; break, \; and, \; or, \; not \; など}
	\item アルファベットの大文字小文字は別の変数として区別される
\end{enumerate}

\begin{figure}[htp]
	\begin{tabular}{cc} \hspace{2truemm}
		\begin{minipage}[ht]{.46\textwidth}
			\begin{tcolorbox}[colframe=blue!75!black,
					colback=blue!10!white,
					coltitle=white, fonttitle=\bfseries,
					title=変数名として使える名前の例]
				a, \;abc, \;ABC,\; a1, \; xxx, \; abc\_xyz, \; \_a
			\end{tcolorbox}
		\end{minipage} \hspace{4truemm}
		\begin{minipage}[ht]{.46\textwidth}
			\begin{tcolorbox}[colframe=red!75!black,
					colback=red!10!white,
					coltitle=white, fonttitle=\bfseries,
					title=変数名として使えない名前の例]
				1a, \; 2b, \;  \$a, \; 1\_, \; if, \; else, \; and, \; not
			\end{tcolorbox}
		\end{minipage}
	\end{tabular}
\end{figure}

\section{代入}
\begin{figure}[htp]
	\begin{tabular}{cc}
		\begin{minipage}[ht]{.25\textwidth}
			\begin{lstlisting}[caption=代入]
a = 5
b = "ABC"
c    \end{lstlisting}
		\end{minipage} \hspace{5truemm}
		\begin{minipage}[ht]{.68\textwidth}
			\begin{screen}
				変数には数値や文字を入れることができる. このことを\textgt{代入}という. \par
				$=$ があるが数学とは\ruby{違}{ちが}い, 左の例では右辺の値(5)を左辺の変数(a)に\par
				代入することとなる. また, 代入をしていない変数は中身がなく,
				この\par 状態のことを\textgt{未定義}という. (左の例では, 3行目の変数 c は未定義) \par
			\end{screen}
		\end{minipage}
	\end{tabular}
\end{figure}

\newpage
\countup{2}
\section*{数式}
\begin{figure}[htp]
	\begin{tabular}{cc}
		\begin{minipage}{.4\textwidth}
			\section{使える記号}
			\begin{tabular}[t]{|c|c|c|c|c|}
				\hline
				記号 & 意味         & 例          & 結果 \\
				\hline \hline
				+    & 加算(足し算) & 5 + 8       & 13   \\ \hline
				-    & 減算(引き算) & 90 - 10     & 80   \\ \hline
				*    & 乗算(掛け算) & 4 * 7       & 28   \\ \hline
				/    & 除算(割り算) & 7 / 2       & 3.5  \\ \hline
				//   & 切り捨て除算 & 7 // 2      & 3    \\ \hline
				$\%$ & 剰余(あまり) & 7 $\%$ 3    & 1    \\ \hline
				**   & 累乗         & 3 ** 4      & 81   \\ \hline
				()   & 括弧(かっこ) & (2 + 4) * 4 & 24   \\ \hline
			\end{tabular}
		\end{minipage} \hspace{20truemm}
		\begin{minipage}{.4\textwidth}
			\section{優先順位}
			\begin{tabular}[t]{|c|c|c|}
				\hline
				優先順位 & 記号 & 意味         \\
				\hline \hline
				1        & ()   & 括弧(かっこ) \\ \hline
				2        & **   & 累乗         \\ \hline
				3        & *    & 乗算(掛け算) \\ \hline
				         & /    & 除算(割り算) \\ \hline
				         & //   & 切り捨て除算 \\ \hline
				         & $\%$ & 剰余(あまり) \\ \hline
				4        & +    & 加算(足し算) \\ \hline
				         & -    & 減算(引き算) \\ \hline
			\end{tabular}
		\end{minipage}
	\end{tabular}
\end{figure}
※優先順位が同じ場合は左から順に計算される. また、演算に使用できる記号はほかにもたくさんある.

\countup{3}
\section*{型(type)}
\section{データの型の種類}
\begin{tabular}[t]{|c|c|c|c|}
	\hline
	型名  & 意味   & 例                           \\
	\hline \hline
	int   & 整数   & 1, -12, 2022, 10, など       \\ \hline
	float & 小数   & 3.14..., 0.5, 12.53, など    \\ \hline
	str   & 文字列 & 'hello',  "こんにちは", など \\ \hline
\end{tabular}
\vspace{2truemm}
\begin{screen}
	数学では整数は小数(実数)の中に含まれるが, コンピューターの世界では小数と整数の\ruby{扱}{あつか}いが異なるので \par 明確に別物.
	文字列は'(シングルクォーテーション), "(ダブルクォーテーション)記号で囲んだ部分のこと.
\end{screen}

\section{型と四則演算}
\begin{itemize}
	\item int 型と float 型の四則演算 \par
	      数学の四則演算と同じ. \par
	      int型とfloat型で四則演算を行うときは, int型をfloat型として型の\ruby{変換}{へんかん}がされ計算が行われる.
	\item str 型と四則演算
	      \begin{itemize}
		      \item str型では数学のような四則演算はできず, 引き算・割り算はできない
		      \item str型同士の足し算は文字列の結合をする(文字列をくっつける) \par
		            (掛け算と違いstr型とint型の足し算はできない) \par
		      \item str型とint型の掛け算は文字列をかけた分だけ\ruby{繰}{く}り返す \par
		            (足し算と違いstr型同士で掛け算はできない)
	      \end{itemize}

	      \begin{figure}[htp]
		      \begin{tabular}{cc} \hspace{5truemm}
			      \begin{minipage}{.4\textwidth}
				      \begin{lstlisting}[caption=str型と四則演算]
# str型の四則演算 (足し算と掛け算)
a = "1"
b = "2"
print(a + b)  # str型の足し算
print(a * 3)  # str型の掛け算 \end{lstlisting}
			      \end{minipage} \hspace{5truemm}
			      \begin{minipage}{.3\textwidth}
				      \info{出力}{12 \par 111}
			      \end{minipage}
		      \end{tabular}
	      \end{figure}

	      \info{説明}{str型の変数aとbが結合されたため出力が 12 となった. \par a * 3で 1 を3回\ruby{繰}{く}り返すため, 出力が 111 となった.}
\end{itemize}

\section{str型の数字}
\begin{figure}[htp]
	\begin{tabular}{cc}
		\begin{minipage}{.45\textwidth}
			\begin{lstlisting}[caption=int型とstr型]
# int型とstr型の出力
int_val = -30    # int_valはint型(整数)
str_val = '-30'  # str_valはstr型(文字列)
print(int_val)
print(str_val)
\end{lstlisting}
		\end{minipage} \hspace{5truemm}
		\begin{minipage}{.3\textwidth}
			\info{出力}{-30 \par -30}
		\end{minipage}
	\end{tabular}
\end{figure}
\info{説明}{出力の見え方は全く同じでも, -30 が int型か str型かという違いがある.}
\newpage

\section*{if文}
\countup{4}
もし(条件式)ならば○○(処理)を実行する, というように条件に合った時だけ ○○(処理)を行うためのもの.
\section{if文の書き方}
\begin{figure}[htp]
	\begin{tabular}{ccc}
		\begin{minipage}[t]{.3\textwidth}
			\begin{lstlisting}[caption=if 文の基本構文-その1]
if 条件式 :
	処理 \end{lstlisting}
		\end{minipage}
		\begin{minipage}[t]{.3\textwidth}
			\begin{lstlisting}[caption=if 文の基本構文-その2]
if 条件式 :
	処理1
else :
	処理2 \end{lstlisting}
		\end{minipage}
		\begin{minipage}[t]{.3\textwidth}
			\begin{lstlisting}[caption=if 文の基本構文-その3]
if 条件式1 :
	処理1
elif 条件式2 :
	処理2
else :
	処理3 \end{lstlisting}
		\end{minipage}
	\end{tabular}
\end{figure}
\vspace{-10truemm}
\begin{figure}[htp]
	\begin{tabular}{ccc} \hspace{3truemm}
		\begin{minipage}[t]{.28\textwidth}
			\begin{itembox}[l]{コード4.4}
				条件が一つのとき
			\end{itembox}
		\end{minipage}
		\begin{minipage}[t]{.3\textwidth}
			\begin{itembox}[l]{コード4.5}
				条件が一つとその条件以外のときに処理をしたいとき
			\end{itembox}
		\end{minipage}
		\begin{minipage}[t]{.33\textwidth}
			\begin{itembox}[l]{コード4.6}
				条件が複数のときとそれらの条件以外のときに処理をしたいとき
			\end{itembox}
		\end{minipage}
	\end{tabular}
\end{figure}
\begin{tcolorbox}[colframe=red!75!black, colback=red!10!white, coltitle=white, fonttitle=\bfseries,
		title=注意]
	\begin{itemize}
		\item 条件式の後ろには必ずコロン(:)をつける
		\item 条件式の後ろには必ずインデント(字下げ)をする
		\item else の後ろには条件式は書かない
	\end{itemize}
\end{tcolorbox}

\section{条件式で使う記号}
\begin{figure}[htp]
	\begin{tabular}{cc}
		\begin{minipage}[c]{.5\textwidth}
			\begin{tabular}[t]{|c|c|c|c|c|}
				\hline
				記号 & 意味       & 例             & 例の意味           \\
				\hline \hline
				$==$ & 等しい     & $x == 5$       & $x$は5と等しい     \\ \hline
				$!=$ & 等しくない & $x$\, $!=$ \,3 & $x$は3と等しくない \\ \hline
				$>$  & より大きい & $5 > 2$        & 5は2より大きい     \\ \hline
				$<$  & より小さい & $2 < 3$        & 2は3より小さい     \\ \hline
				$>=$ & 以上       & $a >= 0$       & $a$は0以上         \\ \hline
				$<=$ & 以下       & $b <= 0$       & $b$は0以下         \\ \hline
			\end{tabular}
		\end{minipage}
		\begin{minipage}[c]{.465\textwidth}
			\begin{screen}
				等しい事を示す記号は, 2つの等号($==$)が \par 使われている.  \par
				1つの等号($=$) では代入になってしまうので注意 \par
				また, 数学では以上・以下は $\geqq$・$\leqq$ と書くが, \par プログラミングではそのように書くことが \par できないため代わりに $>=$・$<=$ と書く.
			\end{screen}
		\end{minipage}
	\end{tabular}
\end{figure}
\newpage

\section{if文の例}
\begin{figure}[htp]
	\begin{tabular}{cc}
		\begin{minipage}[t]{.45\textwidth}
			\begin{lstlisting}[caption=if文の例-その1]
# 変数numの値が2の倍数かどうかの判定
if num % 2 == 0 :
	print("num は2の倍数です")
else :
	print("num は2の倍数ではありません") \end{lstlisting}
		\end{minipage} \hspace{5truemm}
		\begin{minipage}[t]{.45\textwidth}
			\begin{lstlisting}[caption=if文の例-その2]
# 変数numが正の数か負の数か0かどうかの判定
if num > 0 :
	print("num は正の数")
elif num < 0 :
	print("num は負の数")
else :
	print("num は0") \end{lstlisting}
		\end{minipage}
	\end{tabular}
\end{figure}
\vspace{-10truemm}
\begin{figure}[htp]
	\begin{tabular}{cc}
		\begin{minipage}[t]{.46\textwidth}
			\begin{itembox}[l]{コード4.7}
				numの値が2の倍数ならば,  \par "num は2の倍数です" と出力.  \par
				numの値が2の倍数でないならば,  \par "num は2の倍数ではありません" と出力. \par
				2 の倍数の判定は num $\%$ 2 $==$ 0 で行うことができる. \par
			\end{itembox}
		\end{minipage} \hspace{7truemm}
		\begin{minipage}[t]{.44\textwidth}
			\begin{itembox}[l]{コード4.8}
				num が 0 より大きいとき (num $>$ 0) \par "num は正の数" と出力.  \par
				num が 0 より小さいとき (num $<$ 0) \par "num は負の数" と出力.  \par
				それ以外のとき (つまりnum が 0 のとき) \par "num は 0" と出力.  \par
			\end{itembox}
		\end{minipage}
	\end{tabular}
\end{figure}


\section*{複雑なif 文}
\section{if 文の仕組み}
if文の条件式からは True/False という値が返ってくる. これらは bool 型という「分類の値を持つ型」の値である.  \par
条件式があっているときは True
条件式が間違っているときは False \par
if 文が実行されるときは条件式がTrueのときである.
\section{論理演算}
複数の条件があるときに使う.

\subsection{and(論理積)}
複数の条件がすべて合っていてほしいときに使う. \par
すべての条件が合っているときその条件式全体はTrueとなる.

\begin{figure}[htp]
	\begin{tabular}{cc}
		\begin{minipage}{.45\textwidth}
			\begin{lstlisting}[caption=andの例]
if num % 3 == 0 and num % 5 == 0 :
	print("3 と 5 の倍数です")
else :
	print("3 と 5 の倍数でもありません") \end{lstlisting}
		\end{minipage} \hspace{5truemm}
		\begin{minipage}{.45\textwidth}
			\info{説明}{numが3の倍数かつ5の倍数ならば, \par "3 と 5 の倍数です" と出力. \par
				numが3の倍数かつ5の倍数でないならば, \par "3 と 5 の倍数でもありません" と出力.}
		\end{minipage}
	\end{tabular}
\end{figure}

\subsection{or(論理和)}
複数の条件のどれか一つでも合っていてほしいときに使う. \par
どれか一つでも条件が合っているときその条件式全体はTrueとなる.

\begin{figure}[htp]
	\begin{tabular}{cc}
		\begin{minipage}{.45\textwidth}
			\begin{lstlisting}[caption=orの例]
if str == "中学生" or str == "小学生" :
	print("中学生か小学生です") \end{lstlisting}
		\end{minipage} \hspace{5truemm}
		\begin{minipage}{.45\textwidth}
			\info{説明}{strが "中学生" か "小学生" ならば, \par "中学生か小学生です" と出力.}
		\end{minipage}
	\end{tabular}
\end{figure}

\subsection{not(否定)}
True ならば False, False ならば True となる. \par

\begin{figure}[htp]
	\begin{tabular}{cc}
		\begin{minipage}{.45\textwidth}
			\begin{lstlisting}[caption=notの例]
if not (num % 2 == 0) :
	print("奇数です")
else :
	print("偶数です") \end{lstlisting}
		\end{minipage} \hspace{5truemm}
		\begin{minipage}{.46\textwidth}
			\info{説明}{numが偶数でないならば, "奇数です" と出力. \par numが偶数ならば, "偶数です" と出力.}
		\end{minipage}
	\end{tabular}
\end{figure}

\section{計算の優先順位(再び)}
優先順位が高いほど先に計算が行われる. 同じ優先順位の場合は左から右へ順に計算される. (数学の計算と同じ) \par
\begin{tabular}[t]{|c|c|c|}
	\hline
	優先順位 &              & 同じ演算のときの優先順位                    \\
	\hline \hline
	高       & 括弧(かっこ) &                                             \\ \hline
	         & 算術演算     & 累乗(**)                                    \\ \hline
	         &              & 乗算(*),除算(/),切り捨て除算(//),剰余($\%$) \\ \hline
	         &              & 加算(+),減算(-)                             \\ \hline
	         & 比較演算     & $>$, $>=$,  $<$,  $<=$                      \\ \hline
	         & 論理演算     & not(否定)                                   \\ \hline
	         &              & and(論理積)                                 \\ \hline
	低       &              & or(論理和)                                  \\ \hline
\end{tabular} \par
※この表の同じ枠内での優先順位は同じ.

\newpage
\section*{繰り返し文}
\countup{5}
\section{for文}
決められた回数繰り返したいときに使う.
\section{for文の書き方}
\begin{figure}[htp]
	\begin{tabular}{c}
		\begin{minipage}{.35\textwidth}
			\begin{lstlisting}[caption=for文の基本構文]
for 変数名 in 繰り返す回数 :
	繰り返したい処理1
	繰り返したい処理2 \end{lstlisting}
		\end{minipage}
	\end{tabular}
\end{figure}

\begin{tcolorbox}[colframe=red!75!black,
		colback=red!10!white,
		coltitle=white, fonttitle=\bfseries,
		title=注意]
	\begin{itemize}
		\item for文の後ろには必ずコロン(:)をつける
		\item for文の処理はインデント(字下げ)をする (インデントの有無で繰り返す処理の範囲が変わってしまう)
	\end{itemize}
\end{tcolorbox}

\section{for文の例}
\begin{figure}[htp]
	\begin{tabular}{ccc}
		\begin{minipage}[ht]{.42\textwidth}
			\begin{lstlisting}[caption=for文の例]
# 0 から 5 までのすべて足したときの値
s = 0
for i in range(6) :
	print(i)       # 繰り返し部分
	s = s + i  # 繰り返し部分

print(s) # for文の範囲外
\end{lstlisting}
		\end{minipage} \hspace{6truemm}
		\begin{minipage}[ht]{.15\textwidth}
			\info{出力}{0 \par 1 \par 2 \par 3 \par 4 \par 5 \par 15}
		\end{minipage} \hspace{6truemm}
		\begin{minipage}[ht]{.29\textwidth}
			\info{説明}{繰り返し:1回目 \par
				\hspace{13mm}:2回目 \par
				\hspace{13mm}:3回目 \par
				\hspace{13mm}:4回目 \par
				\hspace{13mm}:5回目 \par
				\hspace{13mm}:6回目 \par
				7行目のprint関数の出力}
		\end{minipage}
	\end{tabular}
\end{figure}
\newpage

\section{whlie文}
for文と同じく\ruby{繰}{く}り返しが必要な処理を行うために使うもの.  \par
条件式がTrueの間\ruby{繰}{く}り返しが続き, Falseのとき\ruby{繰}{く}り返しが終了する.

\section{while文の書き方}
\begin{figure}[htp]
	\begin{tabular}{c}
		\begin{minipage}{.55\textwidth}
			\begin{lstlisting}[caption=while文の基本構文]
while 条件式 :
	条件式が'True'のとき繰り返したい処理\end{lstlisting}
		\end{minipage}
	\end{tabular}
\end{figure}

\begin{tcolorbox}[colframe=red!75!black,
		colback=red!10!white,
		coltitle=white, fonttitle=\bfseries,
		title=注意]
	\begin{itemize}
		\item while文の後ろには必ずコロン(:)をつける
		\item while文の処理はインデント(字下げ)をする \par
		      (インデントの有無で繰り返す処理の範囲が変わってしまう)
	\end{itemize}
\end{tcolorbox}

\begin{figure}[htp]
	\begin{tabular}{c}
		\begin{minipage}{.55\textwidth}
			\begin{lstlisting}[caption=while文の注意点]
while True : # もしくは条件式がずっとTrue
	繰り返したい処理  # 処理が永遠に繰り返される \end{lstlisting}
		\end{minipage}
	\end{tabular}
\end{figure}

\info{while文の注意点}{
	while文は条件式がTrueのとき処理を\ruby{繰}{く}り返すので, 「while True:」のときは永遠に処理を\ruby{繰}{く}り返す.
	また, 指定した条件式を間違えたり, 条件式で変数を使っているときに\ruby{繰}{く}り返し処理のなかで変数を条件式を満たすように変化させなかった場合も条件式の結果がTrueだと永遠に処理を\ruby{繰}{く}り返してしまう.}

\begin{figure}[htp]
	\begin{tabular}{c}
		\begin{minipage}{.8\textwidth}
			\begin{lstlisting}[caption=while文の注意点-例1]
# 1 から 10 までの数字を出力 (n への加算操作がないため永遠に繰り返す)
n = 1
while n < 10 :
	print(n) \end{lstlisting}
		\end{minipage}
	\end{tabular}
\end{figure}

\begin{figure}[htp]
	\begin{tabular}{c}
		\begin{minipage}{.8\textwidth}
			\begin{lstlisting}[caption=while文の注意点-例2]
# a が 5 のとき Buzz を出力して終了 (break を忘れているため永遠に繰り返す)
a = 1
while True :
	a = a + 1
	if a == 5 :
		print("Buzz") \end{lstlisting}
		\end{minipage}
	\end{tabular}
\end{figure}

\newpage

\section{break}
for文や while文の\ruby{繰}{く}り返し処理の中で\ruby{繰}{く}り返しを中断したいときに使う. \par
break を使うとfor文や while文の\textgt{\ruby{繰}{く}り返し処理自体が終了}する.
\begin{figure}[h]
	\begin{tabular}{ccc}
		\begin{minipage}[ht]{.3\textwidth}
			\begin{lstlisting}[caption=break の使い方]
n = 0
while True :
  n = n + 1
  print(f'n = {n}')
  if n == 5:
    break    \end{lstlisting}
		\end{minipage} \hspace{5truemm}
		\begin{minipage}[ht]{.15\textwidth}
			\info{出力}{n = 1 \par n = 2 \par n = 3 \par n = 4 \par n = 5}
		\end{minipage} \hspace{5truemm}
		\begin{minipage}[ht]{.4\textwidth}
			\info{説明}{while Trueで永遠に\ruby{繰}{く}り返しが実行 \par されそうだが,
				n が 5 のときに breakによってループが終了するため, 5 まで出力された.  \par}
		\end{minipage}
	\end{tabular}
\end{figure}

\section{continue}
\ruby{繰}{く}り返し処理の中で特定の処理のときだけ, その処理をスキップする場合に使う.  \par
continue を使ってもfor文やwhile文の\textgt{\ruby{繰}{く}り返し自体は終了しない}.
\begin{figure}[h]
	\begin{tabular}{ccc}
		\begin{minipage}[ht]{.3\textwidth}
			\begin{lstlisting}[caption=continue の使い方]
n = 0
while n < 5 :
  n = n + 1
  if n == 3:
    continue
  print(f'n = {n}')  \end{lstlisting}
		\end{minipage} \hspace{5truemm}
		\begin{minipage}[ht]{.15\textwidth}
			\info{出力}{n = 1 \par n = 2 \par n = 3 \par n = 4 \par n = 5}
		\end{minipage} \hspace{5truemm}
		\begin{minipage}[ht]{.4\textwidth}
			\info{説明}{n が 3 のときに continueによって\par ループがスキップされたため, \par
				3 のときだけ結果が出力されなかった.}
		\end{minipage}
	\end{tabular}
\end{figure}

\section{条件式とTrueとbreak}
\begin{figure}[h]
	\begin{tabular}{cc}
		\begin{minipage}[t]{.4\textwidth}
			\begin{lstlisting}[caption=条件式による繰り返しの終了]
n = 0
while n < 5 :
  n = n + 1
  print(f'n = {n}') \end{lstlisting}
		\end{minipage} \hspace{10truemm}
		\begin{minipage}[t]{.4\textwidth}
			\begin{lstlisting}[caption=break による繰り返しの終了]
n = 0
while True :
  n = n + 1
  print(f'n = {n}')
  if n == 5:
    break    \end{lstlisting}
		\end{minipage}
	\end{tabular}
\end{figure}

\begin{tcolorbox}[colframe=gray, colback=black!10!white]
	whileの条件式による終了か break による終了かの違いだけで, どちらも, n が 5 のときに\ruby{繰}{く}り返し処理が \par
	終了する同じ処理である. ただし, breakでの処理の方では, breakを忘れてしまうと処理が永遠に\ruby{繰}{く}り返す\par
	ので注意が必要となる.
\end{tcolorbox}

\newpage

\section*{関数}
\countup{6}
\begin{figure}[htp]
	\begin{tabular}{cc}
		\begin{minipage}[t]{.6\textwidth}
			\section{関数とは}
			関数とは, 決められた処理を実行してその結果を返す \par プログラムの部品のようなもの. \par
			関数には値を入れることができ, 入れる値のことを
			{\textgt {引数 \par (ひきすう)}}と言い,  値を処理して返ってきた結果を \par
			{\textgt {返り値/戻り値(かえりち/もどりち)}}と言う.  \par
			関数にはPythonですでにあるものや自分で作れるものがある.  \par

			関数の引数に関数を入れたり, 関数の返り値を変数に代入する \par ことが出来る.  \par
		\end{minipage}
		\begin{minipage}[t]{.35\textwidth}
			\section{関数を使うときの例}
			\begin{lstlisting}[caption=関数]
# 関数の使い方

関数名(引数) # 引数あり
関数名()    # 引数なし \end{lstlisting}
		\end{minipage}
	\end{tabular}
\end{figure}

\section{Pythonでよく使う関数}
\begin{itemize}
	\item print関数 \par
	      画面に文字列や変数の値を出力する関数. print関数の引数には文字列や変数を入れることができる. \par
	      引数:文字列や数値, 計算結果 \\
	      \begin{itemize}
		      \item[$\circ$] print関数の使い方
			      \begin{figure}[htp]
				      \begin{tabular}{ccc} \hspace{10truemm}
					      \begin{minipage}{.3\textwidth}
						      \begin{lstlisting}[caption=print関数]
print("こんにちは")
print(1000)
a = 10
b = 30
print(a)
print(a + b) \end{lstlisting}
					      \end{minipage} \hspace{3truemm}
					      \begin{minipage}{.17\textwidth}
						      \info{出力}{こんにちは \par 1000 \par 10 \par 40}
					      \end{minipage} \hspace{3truemm}
					      \begin{minipage}{.3\textwidth}
						      \info{説明}{引数の文字列や数値, 変数の値, 計算式の計算結果を出力する}
					      \end{minipage}
				      \end{tabular}
			      \end{figure}

		      \item[$\circ $] print 関数で変数と文字を一緒に使う
			      \begin{figure}[htp]
				      \begin{tabular}{cc} \hspace{10truemm}
					      \begin{minipage}[ht]{.35\textwidth}
						      \begin{lstlisting}[caption=fフォーマット]
name = "タロウ"
print(f'{name}さん、こんにちは')\end{lstlisting}
					      \end{minipage} \hspace{10truemm}
					      \begin{minipage}[ht]{.4\textwidth}
						      \info{出力}{タロウさん、こんにちは}
					      \end{minipage} \hspace{3truemm}
				      \end{tabular}
			      \end{figure}
			      \info{説明}{文字列をf''で囲み (f'文字列' となるように), \{ \} の中に変数を入れることで変数と文字列を \par 一緒に出力することができる.}
	      \end{itemize}
\end{itemize}

\newpage

\begin{itemize}
	\item range関数 \par
	      一定の間隔での数値の並びを生成する関数 \par
	      range関数の引数の与え方は, range(開始値, 終了値, ステップ幅)ある.  \par
	      開始値とステップ幅はそれぞれ省略でき, 省略した場合は(開始値:0, ステップ幅:1)が初期値となる.  \par
	      生成できる値は開始値から{\textgt{終了値より手前の整数}}までの範囲である.

	      \begin{figure}[htp]
		      \begin{tabular}{cc} \hspace{5truemm}
			      \begin{minipage}[ht]{.31\textwidth}
				      \begin{lstlisting}[caption=range関数]
for i in range(5) :
	# 0, 1, 2, 3, 4

for i in range(5, 10) :
	# 5, 6, 7, 8, 9

for i in range(0, 10, 2) :
	# 0, 2, 4, 6, 8
	\end{lstlisting}
			      \end{minipage} \hspace{5truemm}
			      \begin{minipage}[ht]{.58\textwidth}
				      \info{説明}{3 行目では開始値, ステップ幅が省略されているため, 0から\par 始まり, 1ずつ上がっていく\par
					      6 行目では5から始まり, 9までの値を生成する\par
					      9 行目ではステップ幅が2のため, 2ずつ上がり, 終了値10, \par つまり9までの範囲で値を生成する}
			      \end{minipage}
		      \end{tabular}
	      \end{figure}
\end{itemize}

\begin{itemize}
	\item max関数 \par
	      2つ以上の引数でその中から最大の値を返す関数 \par
	      引数:数値, 文字列, リスト \hspace{5truemm}
	      返り値:最大値
	      \begin{figure}[htp]
		      \begin{tabular}{cc} \hspace{5truemm}
			      \begin{minipage}[ht]{.4\textwidth}
				      \begin{lstlisting}[caption=max関数]
max_num = max(1, 3, 4, 2)
max_str = max("abc", "xyz", "ijk")
print(max_num)
print(max_str) \end{lstlisting}
			      \end{minipage} \hspace{5truemm}
			      \begin{minipage}[ht]{.15\textwidth}
				      \info{出力}{4 \par xyz}
			      \end{minipage}
		      \end{tabular}
	      \end{figure}

	      \info{説明}{max\_numには(1, 3, 4, 2)の中で一番大きい値である4が代入される. \par
		      max\_strには("abc", "xyz", "ijk")の中で一番大きい値であるxyzが代入される.}
\end{itemize}

\begin{itemize}
	\item min関数 \par
	      2つ以上の引数でその中から最小の値を返す関数 \par
	      引数:数値, 文字列, リスト \hspace{5mm}
	      返り値:最小値
	      \begin{figure}[htp]
		      \begin{tabular}{cc} \hspace{5truemm}
			      \begin{minipage}[ht]{.4\textwidth}
				      \begin{lstlisting}[caption=min関数]
min_num = min(1, 3, 4, 2)
min_str = min("abc", "xyz", "ijk")
print(min_num)
print(min_str) \end{lstlisting}
			      \end{minipage} \hspace{5truemm}
			      \begin{minipage}[ht]{.15\textwidth}
				      \info{出力}{1 \par abc}
			      \end{minipage}
		      \end{tabular}
	      \end{figure}

	      \info{説明}{min\_numには(1, 3, 4, 2)の中で一番小さい値である1が代入される. \par
		      min\_strには("abc", "xyz", "ijk")の中で一番小さい値であるabcが代入される.}
\end{itemize}
\newpage

\begin{itemize}
	\item input関数 \par
	      変数に任意(好き)な文字列を代入する事が出来る関数 \par
	      引数:なし
	      \begin{figure}[htp]
		      \begin{tabular}{ccc} \hspace{5truemm}
			      \begin{minipage}[ht]{.4\textwidth}
				      \begin{lstlisting}[caption=input関数]
value = input()
print(value) \end{lstlisting}
			      \end{minipage} \hspace{5truemm}
			      \begin{minipage}[ht]{.15\textwidth}
				      \info{入力}{abcdef}
			      \end{minipage} \hspace{5truemm}
			      \begin{minipage}[ht]{.15\textwidth}
				      \info{出力}{abcdef}
			      \end{minipage}
		      \end{tabular}
	      \end{figure}

	      \info{説明}{入力したabcdefがvalueに代入され, print関数で出力される}
\end{itemize}

\begin{itemize}
	\item type関数 \par
	      データの種類を調べる関数 \par
	      引数:データ \hspace{5mm}
	      返り値:種類名
	      \begin{figure}[htp]
		      \begin{tabular}{ccc} \hspace{5truemm}
			      \begin{minipage}[ht]{.28\textwidth}
				      \begin{lstlisting}[caption=type関数]
int_val = 3
float_val = 3.14
print(type(int_val))
print(type(float_val)) \end{lstlisting}
			      \end{minipage} \hspace{2.5truemm}
			      \begin{minipage}[ht]{.2\textwidth}
				      \info{出力}{$<$class 'int'$>$ \par $<$class 'float'$>$ \par}
			      \end{minipage} \hspace{2.5truemm}
			      \begin{minipage}[ht]{.4\textwidth}
				      \info{説明}{int\_valの値はint型, float\_valの値はfloat型とデータの種類が出力された}
			      \end{minipage}
		      \end{tabular}
	      \end{figure}
\end{itemize}

\begin{itemize}
	\item len関数 \par
	      引数の値の長さを返す関数
	      \begin{figure}[htp]
		      \begin{tabular}{ccc} \hspace{5truemm}
			      \begin{minipage}[ht]{.28\textwidth}
				      \begin{lstlisting}[caption=len関数]
s = "abcde"
l = len(s)
print(l) \end{lstlisting}
			      \end{minipage} \hspace{5truemm}
			      \begin{minipage}[ht]{.15\textwidth}
				      \info{出力}{5}
			      \end{minipage} \hspace{5truemm}
			      \begin{minipage}[ht]{.4\textwidth}
				      \info{説明}{変数s の値 "abcde" の長さが5 のため,  \par
					      len関数の戻り値は5 となる}
			      \end{minipage}
		      \end{tabular}
	      \end{figure}
\end{itemize}

\begin{itemize}
	\item sum関数 \par
	      引数の値の合計を返す関数 \par
	      \begin{figure}[htp]
		      \begin{tabular}{ccc} \hspace{5truemm}
			      \begin{minipage}[ht]{.28\textwidth}
				      \begin{lstlisting}[caption=sum関数]
l = [1, 2, 3, 4, 5]
s = sum(l)
print(s) \end{lstlisting}
			      \end{minipage} \hspace{5truemm}
			      \begin{minipage}[ht]{.15\textwidth}
				      \info{出力}{15}
			      \end{minipage} \hspace{5truemm}
			      \begin{minipage}[ht]{.4\textwidth}
				      \info{説明}{変数l の値 [1, 2, 3, 4, 5] の合計が15 のため, sum関数の戻り値は15 となる}
			      \end{minipage}
		      \end{tabular}
	      \end{figure}
\end{itemize}

\newpage

\begin{itemize}
	\item int関数 \par
	      引数に与えられた値をint型に\ruby{変換}{へんかん}する関数
	      \begin{figure}[htp]
		      \begin{tabular}{cc} \hspace{5truemm}
			      \begin{minipage}[ht]{.3\textwidth}
				      \begin{lstlisting}[caption=int関数]
float_val = 3.1415
int_val   = int(float_val)
print(int_val)
print(type(int_val)) \end{lstlisting}
			      \end{minipage} \hspace{5truemm}
			      \begin{minipage}[ht]{.2\textwidth}
				      \info{出力}{3 \par $<$class 'int'$>$}
			      \end{minipage}
		      \end{tabular}
	      \end{figure}
	      \info{説明}{int型は整数型のため小数が整数に\ruby{変換}{へんかん}され, 3.1415 が 3 となった. \par
		      type関数の返り値からもint型になったことが分かる.}
\end{itemize}

\begin{itemize}
	\item float関数 \par
	      引数に与えられた値をfloat型に\ruby{変換}{へんかん}する関数
	      \begin{figure}[htp]
		      \begin{tabular}{cc} \hspace{5truemm}
			      \begin{minipage}[ht]{.3\textwidth}
				      \begin{lstlisting}[caption=float関数]
int_val   = 100
float_val = float(int_val)
print(float_val)
print(type(float_val)) \end{lstlisting}
			      \end{minipage} \hspace{5truemm}
			      \begin{minipage}[ht]{.2\textwidth}
				      \info{出力}{100.0 \par $<$class 'float'$>$}
			      \end{minipage}
		      \end{tabular}
	      \end{figure}
	      \info{説明}{float型は小数の型のため整数が小数に\ruby{変換}{へんかん}され, 100 が 100.0 となった. \par
		      type関数の返り値からもfloat型になったことが分かる.}
\end{itemize}

\begin{itemize}
	\item str関数 \par
	      引数に与えられた値をstr型に\ruby{変換}{へんかん}する関数
	      \begin{figure}[htp]
		      \begin{tabular}{cc} \hspace{5truemm}
			      \begin{minipage}[ht]{.3\textwidth}
				      \begin{lstlisting}[caption=sum関数]
int_val = 100
str_val = str(int_val)
print(str_val)
print(type(str_val)) \end{lstlisting}
			      \end{minipage} \hspace{5truemm}
			      \begin{minipage}[ht]{.2\textwidth}
				      \info{出力}{100 \par $<$class 'str'$>$}
			      \end{minipage}
		      \end{tabular}
	      \end{figure}
	      \info{説明}{見た目ではわからないがint型からstr型に\ruby{変換}{へんかん}されたことが type関数によってわかる}
\end{itemize}

\newpage

\section*{list(リスト)}
\countup{7}
\section{リストとは}
複数の値に順番をつけてまとめて扱うためのデータ型の一つ. \par
\section{リストの生成}
\begin{figure}[htp]
	\begin{tabular}{cc}
		\begin{minipage}[ht]{.3\textwidth}
			\begin{lstlisting}[caption=リストの生成]
a = [5, 1, 3, 4]
b = ['abc', 'def', 'ghi']

t = 5
c = [t, 1, 3, 4]
\end{lstlisting}
		\end{minipage} \hspace{5truemm}
		\begin{minipage}[ht]{.6\textwidth}
			\info{説明}{リストの生成には [ ] の中に {\LARGE ,} で区切り具体的にリストの \par
			中身を書くことで生成できる. \par 4,5 行目のように変数を入れてもよい}
		\end{minipage}
	\end{tabular}
\end{figure}

\section{リストの使い方}
リストの個々のデータのことを\textgt{要素}という言い, 要素の順番を指定する値のことを\textgt{\ruby{添}{そ}え字}という言い方をする.  \par
また, 要素の順番は 0 から始まる.
\subsection{要素へのアクセス方法}
\begin{figure}[htp]
	\begin{tabular}{ccc}
		\begin{minipage}[ht]{.3\textwidth}
			\begin{lstlisting}[caption=要素へのアクセス方法1]
#    要素の順番
#    0  1  2  3
a = [5, 1, 3, 4]
print(a[0])

# 正の添え字と負の添え字
print(a[3], a[-1]) \end{lstlisting}
		\end{minipage} \hspace{3truemm}
		\begin{minipage}[ht]{.15\textwidth}
			\info{出力}{5 \par 4 4}
		\end{minipage} \hspace{3truemm}
		\begin{minipage}[ht]{.46\textwidth}
			\info{説明}{リストの 0 番目である値 5 が出力される. \par
				添え字には負の値を指定することができ, \par 負の値を指定した場合はリストの後ろの \par 要素から指定されていく.}
		\end{minipage}
	\end{tabular}
\end{figure}

\begin{figure}[htp]
	\begin{tabular}{ccc}
		\begin{minipage}[ht]{.3\textwidth}
			\begin{lstlisting}[caption=要素へのアクセス方法2]
#    要素の順番
#    0  1  2  3
a = [5, 1, 3, 4]
print(a[0:3]) \end{lstlisting}
		\end{minipage} \hspace{3truemm}
		\begin{minipage}[ht]{.15\textwidth}
			\info{出力}{[5, 1, 3]}
		\end{minipage}
	\end{tabular}
	\info{説明}{添え字に[先頭番号:終了番号]を与えると, リストの一部だけを取り出すことができる. \par
	このとき, 終了番号の手前までの値が取り出せることに注意する.
	このような添字の指定の仕方を\textgt{スライス}という. \par}
\end{figure}

\subsection{リストの基本操作}
\begin{itemize}
	\item 要素の追加 \par
	      リストの末尾に要素を追加する
	      \begin{figure}[htp]
		      \begin{tabular}{ccc}  \hspace{5truemm}
			      \begin{minipage}[ht]{.3\textwidth}
				      \begin{lstlisting}[caption=要素の追加]
a = [1, 2, 3, 4]
a.append(5)
print(a) \end{lstlisting}
			      \end{minipage} \hspace{5truemm}
			      \begin{minipage}[ht]{.25\textwidth}
				      \info{出力}{[1, 2, 3, 4, 5]}
			      \end{minipage}
		      \end{tabular}
	      \end{figure}
	      \info{説明}{リストの末尾に指定した値が追加される.
	      append()を使うとき()の中は{\textgt{追加したい値}}を書く. \par
	      リストへの追加はappend()以外にもinsert()関数を使うことで任意の位置に追加することができる.}
\end{itemize}

\begin{itemize}
	\item 要素の削除 \par
	      リストの末尾の要素を削除する
	      \begin{figure}[htp]
		      \begin{tabular}{cc}  \hspace{5truemm}
			      \begin{minipage}[ht]{.3\textwidth}
				      \begin{lstlisting}[caption=要素の削除]
a = [1, 2, 3, 4]
a.pop()
print(a)
a.pop(0)
print(a) \end{lstlisting}
			      \end{minipage} \hspace{5truemm}
			      \begin{minipage}[ht]{.25\textwidth}
				      \info{出力}{[1, 2, 3] \par [2, 3]}
			      \end{minipage}
		      \end{tabular}
	      \end{figure}

	      \info{説明}{引数を指定しない場合はリストの末尾を, 指定した場合は指定した添え字の値が削除される. \par
		      pop()を使うときは()の中は, {\textgt{削除したい値の添え字}}にする.}
\end{itemize}

\newpage

\section{スライス}
リストの一部を取り出すことができる. \par
スライスの書き方は[先頭番号:終了番号]である. \par
このとき, 終了番号の手前までの値が取り出せることに注意する. \par
\begin{figure}[htp]
	\begin{tabular}{ccc}
		\begin{minipage}[ht]{.3\textwidth}
			\begin{lstlisting}[caption=スライス]
a = [1, 2, 3, 4, 5]
print(a[0:3])
print(a[1:2:4])
print(a[2:])
print(a[:3]) \end{lstlisting}
		\end{minipage} \hspace{3truemm}
		\begin{minipage}[ht]{.15\textwidth}
			\info{出力}{[1, 2, 3] \par [2, 3, 4] \par [3, 4, 5] \par [1, 2, 3]}
		\end{minipage}
	\end{tabular}
	\info{説明}{1行目ではリストの0番目から2番目までの値が取り出されている. \par
		2行目ではリストの1番目から3番目までの値が取り出されている. \par
		3行目ではリストの2番目から最後までの値が取り出されている. \par
		4行目ではリストの0番目から2番目までの値が取り出されている.}
\end{figure}

\section*{自作関数}
\countup{8}
\section{自作関数とは}
自分で関数を作ることができる. \par
関数を作ることで, 何度も同じ処理を書く必要がなくなり, プログラムの見通しが良くなる. \par
\section{自作関数の書き方}
\begin{figure}[htp]
	\begin{tabular}{c}
		\begin{minipage}[ht]{.6\textwidth}
			\begin{lstlisting}[caption=自作関数の書き方]
def 関数名(引数) :
	実行したいプログラム
	return 返り値        # 返り値がある場合 \end{lstlisting}
		\end{minipage}
	\end{tabular}

	\begin{itembox}[l]{関数を作るときのルール}
		\begin{itemize}
			\item 複数の関数を作るときは, 関数名が被らないように注意をすること.  関数名のルールは変数名の時と同じ.
			\item 引数は2つ以上でもよい. そのときは {\textgt {,}}で引数と引数を区切る. \par
			      また, 引数はなくてもよい. しかし, その場合でも関数名の後ろには()をつけなければいけない
			\item return 文は関数定義の最後以外の場所, if文の中などに書いてもよい. \par 返り値がない, いらない場合はreturn 文を書かなくてもよい.
			\item 関数の返り値には計算式を書いてもよい.
			\item 関数を作る時の def 後の( )内の引数名と実際に呼び出して使う時の引数名は一致している必要はない.
			\item 関数の中身はインデントを揃えて書く.
		\end{itemize}
	\end{itembox}
\end{figure}

\section*{スコープ}
\countup{9}
\section{スコープとは}
スコープとは, 変数が有効な範囲のことである. スコープには大きく分けて2つの種類がある. \par
\begin{itemize}
	\item {ローカル変数 :限られた範囲で使われる変数} \par
	      関数内で定義された(代入された)変数は関数内でのみ利用可能で, 関数の実行ごとに関数の実行が終了すれば失われる

	\item {グローバル変数:プログラム全体で使える変数} \par
	      関数外で定義されている変数は値を読み取ることのみ可能.
	      関数内で global 宣言された変数のみ, グローバル変数に代入可能.
	      このglobal宣言をせずに代入をするとグローバル変数がローカル変数になってしまう.
\end{itemize}

\section*{モジュール}
\countup{10}
\section{モジュールとは}
モジュールとは, Pythonでよく使われる関数や定数をまとめたもの. \par
モジュールを使うことで, そのモジュールに定義されている関数や定数を使うことができる. \par
モジュールも自作することができる.
\section{モジュールのインポート方法}
モジュールのインポート方法は大きく分けて以下の2つ
\begin{itemize}
	\item{import モジュール名}
	\item{from モジュール名 import 機能名}
\end{itemize}
\subsection{import と from について}
\begin{itemize}
	\item{import~} \par
	モジュール全体を利用する, という宣言 \par
	モジュールの関数を使うときは, 関数の前にモジュール名を付ける必要がある.
	\item{from~} \par
	モジュールの一部を利用する, という宣言 \par
	1つ1つ呼び出す代わりに関数を使うときに関数の前にモジュール名は付けない. \par
	from モジュール名 import * はそのモジュールの関数や変数をすべて直接呼び出している.
\end{itemize}

\info{import~と from~の使い分け}{基本は自由. 自分やチームのルールがある場合はそれに従って書く. \par
	変数名や関数名はかぶる可能性が有るため, 少数しかモジュールや関数を使わないなら from~を \par
	たくさんのモジュールを使うときは import~を使う方がいい場合が多い.}

\newpage

\section{自作モジュールを使用したインポートの紹介}
自作のモジュールを使うときは, モジュールのファイルが使いたいプログラムのファイルと同じ場所にあり, \par モジュール名と同じファイルがないように気を付ける.  \par
次のような自作モジュール(sample.py)があるとする. このときの, インポート方法とモジュールの変数へ\par のアクセス方法を紹介する. \par

\begin{figure}[htp]
	\begin{tabular}{c}
		\begin{minipage}{.4\textwidth}
			\begin{lstlisting}[caption=sample.py]
hello = "こんにちは"
color = ["red", "green", "黒"]

def right(var_list) :
	ret = var_list[-1]
	print(ret)
	return ret \end{lstlisting}
		\end{minipage}
	\end{tabular}
\end{figure}


\begin{figure}[htp]
	\begin{tabular}{cc}
		\begin{minipage}[ht]{.3\textwidth}
			\begin{lstlisting}[caption=インポートの仕方1]
import sample

sample.color
sample.hello \end{lstlisting}
		\end{minipage} \hspace{5truemm}
		\begin{minipage}[ht]{.6\textwidth}
			\info{import モジュール名}{モジュール名. というのを頭につけて変数や関数の名前を書く. \par
				今回であれば sample. をつける必要がある.}
		\end{minipage}
	\end{tabular}
\end{figure}

\begin{figure}[htp]
	\begin{tabular}{cc}
		\begin{minipage}[ht]{.3\textwidth}
			\begin{lstlisting}[caption=インポートの仕方2]
import sample as sp

sp.color
sp.hello \end{lstlisting}
		\end{minipage} \hspace{5truemm}
		\begin{minipage}[ht]{.6\textwidth}
			\info{import モジュール名 as 別の名前}{モジュール名を別の名前に変更して, 変更した名前を頭につけて変数や関数の名前を書く.
				今回であれば sample. ではなく sp. をつける必要がある.  \par
				この方法はモジュール名が長いものを省略して書きたいときによく使われる方法である.}
		\end{minipage}
	\end{tabular}
\end{figure}
\newpage

\begin{figure}[htp]
	\begin{tabular}{c}
		\begin{minipage}[ht]{.4\textwidth}
			\begin{lstlisting}[caption=インポートの仕方3]
from sample import color, right

color
hello \end{lstlisting}
		\end{minipage}
	\end{tabular} \hspace{15truemm}
\end{figure}

\begin{figure}[htp] \vspace{-6truemm}
	\begin{tabular}{c} \hspace{3truemm}
		\begin{minipage}[ht]{.915\textwidth}
			\info{from モジュール名 import モジュールの関数や変数(複数可)}{モジュール名. というのを頭につけて関数の名前や変数を書く必要がなくなる. \par
				しかし, 宣言していない関数や変数は使えないため注意 (今回では hello は宣言していないため\par 使えない)}
		\end{minipage}
	\end{tabular}
\end{figure}

\begin{figure}[htp]
	\begin{tabular}{cc}
		\begin{minipage}[ht]{.3\textwidth}
			\begin{lstlisting}[caption=インポートの仕方4]
from sample import *

color
hello \end{lstlisting}
		\end{minipage} \hspace{5truemm}
		\begin{minipage}[ht]{.6\textwidth}
			\info{from モジュール名 import *}{モジュールの関数や変数を import の後ろに\par 書かなくてもモジュールの関数や変数が使えるようになる.}
		\end{minipage}
	\end{tabular}
\end{figure}

\end{document}