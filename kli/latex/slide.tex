% https://dreamer-uma.com/latex-beamer/
%%%%%%%%%% Beamerの初期設定 %%%%%%%%%%%
% beamerを使用する初期設定
\documentclass[aspectratio=169, dvipdfmx, 12pt]{beamer}

% 使用するパッケージ
\usepackage{here, amsmath, latexsym, amssymb, bm, ascmac, mathtools, multicol, tcolorbox, subfig}

%デザインの選択 (省略可)
\usetheme{Boadilla}
%%%%%%%%%%%%%%%%%%%%%%%%%%%%%%%%%%%%%


%%%%%%%%%% Beamerの基本的なコード %%%%%%%%%%
% 属性
\title{タイトル}
\subtitle{サブタイトル}
\author[著者略称]{作者}
\institute[所属略称]{所属}
\date{\today}

% スライドの始まり
\begin{document}

% タイトルページ
\frame{\maketitle}

% スライド Example
\begin{frame}{スライド}
	数式を書くことができるよ。
	\begin{equation}
		\frac{1}{s^{2}}\frac{\partial^{2} u}{\partial t^{2}} = \frac{\partial^{2} u}{\partial x^{2}} + \frac{\partial^{2} u}{\partial y^{2}} + \frac{\partial^{2} u}{\partial z^{2}}
	\end{equation}
\end{frame}

% 目次ページ
\begin{frame}{目次}
	\tableofcontents
\end{frame}

% 目次の具体例
\section{目次の具体例}
\begin{frame}{スライド}
	arrayも使えます.
	\begin{align}
		x & = a + 3 \\
		y & = b -4
	\end{align}
\end{frame}

% 箇条書き
\section{箇条書き}
\begin{frame}{箇条書き}
	\begin{itemize}
		\item item 1
		      \begin{enumerate}
			      \item item1-1
			      \item item1-2
		      \end{enumerate}
		\item item 2
		      \begin{enumerate}[I]
			      \item item 2-1
			      \item item 2-2
			      \item item 2-3
		      \end{enumerate}
	\end{itemize}
\end{frame}

% ブロック環境
\section{ブロック環境}
\begin{frame}{ブロック環境}
	% 通常のブロック
	\begin{block}{block}
		simple block
	\end{block}
	% 注意のブロック
	\begin{alertblock}{alertblock}
		alertblock
	\end{alertblock}
	% Example ブロック
	\begin{exampleblock}{exampleblock}
		exampleblock
	\end{exampleblock}
\end{frame}

% 数学ブロック環境
\begin{frame}{数学ブロック環境}
	\begin{theorem}[定理名]
		$e^{ix} = \cos x + i \sin x$
	\end{theorem}
	\begin{definition}[定義名]
		$e^{x} \approx 1 + x$
	\end{definition}
	\begin{corollary}[系名]
		$a + b + c = 0$
	\end{corollary}
	\begin{proof}
		$\int_{-\infty}^{\infty} e^{-a x^{2}} = 2 \int_{0}^{\infty} e^{-a x^{2}}$
	\end{proof}
\end{frame}

% 表スライド
\section{表}
\begin{frame}{表の追加}
	\begin{table}
		\caption{Caption}
		\label{table:sample}
		\centering
		\begin{tabular}{ccc}
			\hline
			料理   & 値段  & 場所 \\
			\hline \hline
			チキン & 200円 & 公園 \\
			ピザ   & 300円 & 公園 \\
			ご飯   & 100円 & 室内 \\
			パン   & 70円  & 室内 \\
			\hline
		\end{tabular}
	\end{table}
\end{frame}
\end{document}
